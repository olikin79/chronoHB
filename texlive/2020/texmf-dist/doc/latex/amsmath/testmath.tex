%%% ====================================================================
%%% @LaTeX-file{
%%%   filename  = "testmath.tex",
%%%   version   = "2.0",
%%%   date      = "1999/11/15",
%%%   time      = "15:09:17 EST",
%%%   checksum  = "07762 2342 7811 82371",
%%%   author    = "American Mathematical Society",
%%%   copyright = "Copyright 1995, 1999 American Mathematical Society,
%%%                all rights reserved.  Copying of this file is
%%%                authorized only if either:
%%%                (1) you make absolutely no changes to your copy,
%%%                including name; OR
%%%                (2) if you do make changes, you first rename it
%%%                to some other name.",
%%%   address   = "American Mathematical Society,
%%%                Technical Support,
%%%                Electronic Products and Services,
%%%                P. O. Box 6248,
%%%                Providence, RI 02940,
%%%                USA",
%%%   telephone = "401-455-4080 or (in the USA and Canada)
%%%                800-321-4AMS (321-4267)",
%%%   FAX       = "401-331-3842",
%%%   email     = "tech-support@ams.org (Internet)",
%%%   codetable = "ISO/ASCII",
%%%   keywords  = "latex, amsmath, examples, documentation",
%%%   supported = "yes",
%%%   abstract  = "This is a test file containing extensive examples of
%%%                mathematical constructs supported by the amsmath
%%%                package.",
%%%   docstring = "The checksum field above contains a CRC-16
%%%                checksum as the first value, followed by the
%%%                equivalent of the standard UNIX wc (word
%%%                count) utility output of lines, words, and
%%%                characters.  This is produced by Robert
%%%                Solovay's checksum utility.",
%%% }
%%% ====================================================================
\NeedsTeXFormat{LaTeX2e}% LaTeX 2.09 can't be used (nor non-LaTeX)
[1994/12/01]% LaTeX date must December 1994 or later
\documentclass[draft]{article}
\pagestyle{headings}

\title{Sample Paper for the \pkg{amsmath} Package\\
File name: \fn{testmath.tex}}
\author{American Mathematical Society}
\date{Version 2.0, 1999/11/15}

\usepackage{amsmath,amsthm}

%    Some definitions useful in producing this sort of documentation:
\chardef\bslash=`\\ % p. 424, TeXbook
%    Normalized (nonbold, nonitalic) tt font, to avoid font
%    substitution warning messages if tt is used inside section
%    headings and other places where odd font combinations might
%    result.
\newcommand{\ntt}{\normalfont\ttfamily}
%    command name
\newcommand{\cn}[1]{{\protect\ntt\bslash#1}}
%    LaTeX package name
\newcommand{\pkg}[1]{{\protect\ntt#1}}
%    File name
\newcommand{\fn}[1]{{\protect\ntt#1}}
%    environment name
\newcommand{\env}[1]{{\protect\ntt#1}}
\hfuzz1pc % Don't bother to report overfull boxes if overage is < 1pc

%       Theorem environments

%% \theoremstyle{plain} %% This is the default
\newtheorem{thm}{Theorem}[section]
\newtheorem{cor}[thm]{Corollary}
\newtheorem{lem}[thm]{Lemma}
\newtheorem{prop}[thm]{Proposition}
\newtheorem{ax}{Axiom}

\theoremstyle{definition}
\newtheorem{defn}{Definition}[section]

\theoremstyle{remark}
\newtheorem{rem}{Remark}[section]
\newtheorem*{notation}{Notation}

%\numberwithin{equation}{section}

\newcommand{\thmref}[1]{Theorem~\ref{#1}}
\newcommand{\secref}[1]{\S\ref{#1}}
\newcommand{\lemref}[1]{Lemma~\ref{#1}}

\newcommand{\bysame}{\mbox{\rule{3em}{.4pt}}\,}

%       Math definitions

\newcommand{\A}{\mathcal{A}}
\newcommand{\B}{\mathcal{B}}
\newcommand{\st}{\sigma}
\newcommand{\XcY}{{(X,Y)}}
\newcommand{\SX}{{S_X}}
\newcommand{\SY}{{S_Y}}
\newcommand{\SXY}{{S_{X,Y}}}
\newcommand{\SXgYy}{{S_{X|Y}(y)}}
\newcommand{\Cw}[1]{{\hat C_#1(X|Y)}}
\newcommand{\G}{{G(X|Y)}}
\newcommand{\PY}{{P_{\mathcal{Y}}}}
\newcommand{\X}{\mathcal{X}}
\newcommand{\wt}{\widetilde}
\newcommand{\wh}{\widehat}

\DeclareMathOperator{\per}{per}
\DeclareMathOperator{\cov}{cov}
\DeclareMathOperator{\non}{non}
\DeclareMathOperator{\cf}{cf}
\DeclareMathOperator{\add}{add}
\DeclareMathOperator{\Cham}{Cham}
\DeclareMathOperator{\IM}{Im}
\DeclareMathOperator{\esssup}{ess\,sup}
\DeclareMathOperator{\meas}{meas}
\DeclareMathOperator{\seg}{seg}

%    \interval is used to provide better spacing after a [ that
%    is used as a closing delimiter.
\newcommand{\interval}[1]{\mathinner{#1}}

%    Notation for an expression evaluated at a particular condition. The
%    optional argument can be used to override automatic sizing of the
%    right vert bar, e.g. \eval[\biggr]{...}_{...}
\newcommand{\eval}[2][\right]{\relax
  \ifx#1\right\relax \left.\fi#2#1\rvert}

%    Enclose the argument in vert-bar delimiters:
\newcommand{\envert}[1]{\left\lvert#1\right\rvert}
\let\abs=\envert

%    Enclose the argument in double-vert-bar delimiters:
\newcommand{\enVert}[1]{\left\lVert#1\right\rVert}
\let\norm=\enVert

\begin{document}
\maketitle
\markboth{Sample paper for the {\protect\ntt\lowercase{amsmath}} package}
{Sample paper for the {\protect\ntt\lowercase{amsmath}} package}
\renewcommand{\sectionmark}[1]{}

\section{Introduction}

This paper contains examples of various features from \AmS-\LaTeX{}.

\section{Enumeration of Hamiltonian paths in a graph}

Let $\mathbf{A}=(a_{ij})$ be the adjacency matrix of graph $G$. The
corresponding Kirchhoff matrix $\mathbf{K}=(k_{ij})$ is obtained from
$\mathbf{A}$ by replacing in $-\mathbf{A}$ each diagonal entry by the
degree of its corresponding vertex; i.e., the $i$th diagonal entry is
identified with the degree of the $i$th vertex. It is well known that
\begin{equation}
\det\mathbf{K}(i|i)=\text{ the number of spanning trees of $G$},
\quad i=1,\dots,n
\end{equation}
where $\mathbf{K}(i|i)$ is the $i$th principal submatrix of
$\mathbf{K}$.
\begin{verbatim}
\det\mathbf{K}(i|i)=\text{ the number of spanning trees of $G$},
\end{verbatim}

Let $C_{i(j)}$ be the set of graphs obtained from $G$ by attaching edge
$(v_iv_j)$ to each spanning tree of $G$. Denote by $C_i=\bigcup_j
C_{i(j)}$. It is obvious that the collection of Hamiltonian cycles is a
subset of $C_i$. Note that the cardinality of $C_i$ is $k_{ii}\det
\mathbf{K}(i|i)$. Let $\wh X=\{\hat x_1,\dots,\hat x_n\}$.
\begin{verbatim}
$\wh X=\{\hat x_1,\dots,\hat x_n\}$
\end{verbatim}
Define multiplication for the elements of $\wh X$ by
\begin{equation}\label{multdef}
\hat x_i\hat x_j=\hat x_j\hat x_i,\quad \hat x^2_i=0,\quad
i,j=1,\dots,n.
\end{equation}
Let $\hat k_{ij}=k_{ij}\hat x_j$ and $\hat k_{ij}=-\sum_{j\not=i} \hat
k_{ij}$. Then the number of Hamiltonian cycles $H_c$ is given by the
relation \cite{liuchow:formalsum}
\begin{equation}\label{H-cycles}
\biggl(\prod^n_{\,j=1}\hat x_j\biggr)H_c=\frac{1}{2}\hat k_{ij}\det
\wh{\mathbf{K}}(i|i),\qquad i=1,\dots,n.
\end{equation}
The task here is to express \eqref{H-cycles}
in a form free of any $\hat x_i$,
$i=1,\dots,n$. The result also leads to the resolution of enumeration of
Hamiltonian paths in a graph.

It is well known that the enumeration of Hamiltonian cycles and paths in
a complete graph $K_n$ and in a complete bipartite graph $K_{n_1n_2}$
can only be found from \textit{first combinatorial principles}
\cite{hapa:graphenum}. One wonders if there exists a formula which can
be used very efficiently to produce $K_n$ and $K_{n_1n_2}$. Recently,
using Lagrangian methods, Goulden and Jackson have shown that $H_c$ can
be expressed in terms of the determinant and permanent of the adjacency
matrix \cite{gouja:lagrmeth}. However, the formula of Goulden and
Jackson determines neither $K_n$ nor $K_{n_1n_2}$ effectively. In this
paper, using an algebraic method, we parametrize the adjacency matrix.
The resulting formula also involves the determinant and permanent, but
it can easily be applied to $K_n$ and $K_{n_1n_2}$. In addition, we
eliminate the permanent from $H_c$ and show that $H_c$ can be
represented by a determinantal function of multivariables, each variable
with domain $\{0,1\}$. Furthermore, we show that $H_c$ can be written by
number of spanning trees of subgraphs. Finally, we apply the formulas to
a complete multigraph $K_{n_1\dots n_p}$.

The conditions $a_{ij}=a_{ji}$, $i,j=1,\dots,n$, are not required in
this paper. All formulas can be extended to a digraph simply by
multiplying $H_c$ by 2.

\section{Main Theorem}
\label{s:mt}

\begin{notation} For $p,q\in P$ and $n\in\omega$ we write
$(q,n)\le(p,n)$ if $q\le p$ and $A_{q,n}=A_{p,n}$.
\begin{verbatim}
\begin{notation} For $p,q\in P$ and $n\in\omega$
...
\end{notation}
\end{verbatim}
\end{notation}

Let $\mathbf{B}=(b_{ij})$ be an $n\times n$ matrix. Let $\mathbf{n}=\{1,
\dots,n\}$. Using the properties of \eqref{multdef}, it is readily seen
that

\begin{lem}\label{lem-per}
\begin{equation}
\prod_{i\in\mathbf{n}}
\biggl(\sum_{\,j\in\mathbf{n}}b_{ij}\hat x_i\biggr)
=\biggl(\prod_{\,i\in\mathbf{n}}\hat x_i\biggr)\per \mathbf{B}
\end{equation}
where $\per \mathbf{B}$ is the permanent of $\mathbf{B}$.
\end{lem}

Let $\wh Y=\{\hat y_1,\dots,\hat y_n\}$. Define multiplication
for the elements of $\wh Y$ by
\begin{equation}
\hat y_i\hat y_j+\hat y_j\hat y_i=0,\quad i,j=1,\dots,n.
\end{equation}
Then, it follows that
\begin{lem}\label{lem-det}
\begin{equation}\label{detprod}
\prod_{i\in\mathbf{n}}
\biggl(\sum_{\,j\in\mathbf{n}}b_{ij}\hat y_j\biggr)
=\biggl(\prod_{\,i\in\mathbf{n}}\hat y_i\biggr)\det\mathbf{B}.
\end{equation}
\end{lem}

Note that all basic properties of determinants are direct consequences
of Lemma ~\ref{lem-det}. Write
\begin{equation}\label{sum-bij}
\sum_{j\in\mathbf{n}}b_{ij}\hat y_j=\sum_{j\in\mathbf{n}}b^{(\lambda)}
_{ij}\hat y_j+(b_{ii}-\lambda_i)\hat y_i\hat y
\end{equation}
where
\begin{equation}
b^{(\lambda)}_{ii}=\lambda_i,\quad b^{(\lambda)}_{ij}=b_{ij},
\quad i\not=j.
\end{equation}
Let $\mathbf{B}^{(\lambda)}=(b^{(\lambda)}_{ij})$. By \eqref{detprod}
and \eqref{sum-bij}, it is
straightforward to show the following
result:
\begin{thm}\label{thm-main}
\begin{equation}\label{detB}
\det\mathbf{B}=
\sum^n_{l =0}\sum_{I_l \subseteq n}
\prod_{i\in I_l}(b_{ii}-\lambda_i)
\det\mathbf{B}^{(\lambda)}(I_l |I_l ),
\end{equation}
where $I_l =\{i_1,\dots,i_l \}$ and $\mathbf{B}^{(\lambda)}(I_l |I_l )$
is the principal submatrix obtained from $\mathbf{B}^{(\lambda)}$
by deleting its $i_1,\dots,i_l $ rows and columns.
\end{thm}

\begin{rem}
Let $\mathbf{M}$ be an $n\times n$ matrix. The convention
$\mathbf{M}(\mathbf{n}|\mathbf{n})=1$ has been used in \eqref{detB} and
hereafter.
\end{rem}

Before proceeding with our discussion, we pause to note that
\thmref{thm-main} yields immediately a fundamental formula which can be
used to compute the coefficients of a characteristic polynomial
\cite{mami:matrixth}:
\begin{cor}\label{BI}
Write $\det(\mathbf{B}-x\mathbf{I})=\sum^n_{l =0}(-1)
^l b_l x^l $. Then
\begin{equation}\label{bl-sum}
b_l =\sum_{I_l \subseteq\mathbf{n}}\det\mathbf{B}(I_l |I_l ).
\end{equation}
\end{cor}
Let
\begin{equation}
\mathbf{K}(t,t_1,\dots,t_n)
=\begin{pmatrix} D_1t&-a_{12}t_2&\dots&-a_{1n}t_n\\
-a_{21}t_1&D_2t&\dots&-a_{2n}t_n\\
\hdotsfor[2]{4}\\
-a_{n1}t_1&-a_{n2}t_2&\dots&D_nt\end{pmatrix},
\end{equation}
\begin{verbatim}
\begin{pmatrix} D_1t&-a_{12}t_2&\dots&-a_{1n}t_n\\
-a_{21}t_1&D_2t&\dots&-a_{2n}t_n\\
\hdotsfor[2]{4}\\
-a_{n1}t_1&-a_{n2}t_2&\dots&D_nt\end{pmatrix}
\end{verbatim}
where
\begin{equation}
D_i=\sum_{j\in\mathbf{n}}a_{ij}t_j,\quad i=1,\dots,n.
\end{equation}

Set
\begin{equation*}
D(t_1,\dots,t_n)=\frac{\delta}{\delta t}\eval{\det\mathbf{K}(t,t_1,\dots,t_n)
}_{t=1}.
\end{equation*}
Then
\begin{equation}\label{sum-Di}
D(t_1,\dots,t_n)
=\sum_{i\in\mathbf{n}}D_i\det\mathbf{K}(t=1,t_1,\dots,t_n; i|i),
\end{equation}
where $\mathbf{K}(t=1,t_1,\dots,t_n; i|i)$ is the $i$th principal
submatrix of $\mathbf{K}(t=1,t_1,\dots,t_n)$.

Theorem ~\ref{thm-main} leads to
\begin{equation}\label{detK1}
\det\mathbf{K}(t_1,t_1,\dots,t_n)
=\sum_{I\in\mathbf{n}}(-1)^{\envert{I}}t^{n-\envert{I}}
\prod_{i\in I}t_i\prod_{j\in I}(D_j+\lambda_jt_j)\det\mathbf{A}
^{(\lambda t)}(\overline{I}|\overline I).
\end{equation}
Note that
\begin{equation}\label{detK2}
\det\mathbf{K}(t=1,t_1,\dots,t_n)=\sum_{I\in\mathbf{n}}(-1)^{\envert{I}}
\prod_{i\in I}t_i\prod_{j\in I}(D_j+\lambda_jt_j)\det\mathbf{A}
^{(\lambda)}(\overline{I}|\overline{I})=0.
\end{equation}

Let $t_i=\hat x_i,i=1,\dots,n$. Lemma ~\ref{lem-per} yields
\begin{multline}
\biggl(\sum_{\,i\in\mathbf{n}}a_{l _i}x_i\biggr)
\det\mathbf{K}(t=1,x_1,\dots,x_n;l |l )\\
=\biggl(\prod_{\,i\in\mathbf{n}}\hat x_i\biggr)
\sum_{I\subseteq\mathbf{n}-\{l \}}
(-1)^{\envert{I}}\per\mathbf{A}^{(\lambda)}(I|I)
\det\mathbf{A}^{(\lambda)}
(\overline I\cup\{l \}|\overline I\cup\{l \}).
\label{sum-ali}
\end{multline}
\begin{verbatim}
\begin{multline}
\biggl(\sum_{\,i\in\mathbf{n}}a_{l _i}x_i\biggr)
\det\mathbf{K}(t=1,x_1,\dots,x_n;l |l )\\
=\biggl(\prod_{\,i\in\mathbf{n}}\hat x_i\biggr)
\sum_{I\subseteq\mathbf{n}-\{l \}}
(-1)^{\envert{I}}\per\mathbf{A}^{(\lambda)}(I|I)
\det\mathbf{A}^{(\lambda)}
(\overline I\cup\{l \}|\overline I\cup\{l \}).
\label{sum-ali}
\end{multline}
\end{verbatim}

By \eqref{H-cycles}, \eqref{detprod}, and \eqref{sum-bij}, we have
\begin{prop}\label{prop:eg}
\begin{equation}
H_c=\frac1{2n}\sum^n_{l =0}(-1)^{l}
D_{l},
\end{equation}
where
\begin{equation}\label{delta-l}
D_{l}=\eval[2]{\sum_{I_{l}\subseteq \mathbf{n}}
D(t_1,\dots,t_n)}_{t_i=\left\{\begin{smallmatrix}
0,& \text{if }i\in I_{l}\quad\\% \quad added for centering
1,& \text{otherwise}\end{smallmatrix}\right.\;,\;\; i=1,\dots,n}.
\end{equation}
\end{prop}

\section{Application}
\label{lincomp}

We consider here the applications of Theorems~\ref{th-info-ow-ow} and
~\ref{th-weak-ske-owf} to a complete
multipartite graph $K_{n_1\dots n_p}$. It can be shown that the
number of spanning trees of $K_{n_1\dots n_p}$
may be written
\begin{equation}\label{e:st}
T=n^{p-2}\prod^p_{i=1}
(n-n_i)^{n_i-1}
\end{equation}
where
\begin{equation}
n=n_1+\dots+n_p.
\end{equation}

It follows from Theorems~\ref{th-info-ow-ow} and
~\ref{th-weak-ske-owf} that
\begin{equation}\label{e:barwq}
\begin{split}
H_c&=\frac1{2n}
\sum^n_{{l}=0}(-1)^{l}(n-{l})^{p-2}
\sum_{l _1+\dots+l _p=l}\prod^p_{i=1}
\binom{n_i}{l _i}\\
&\quad\cdot[(n-l )-(n_i-l _i)]^{n_i-l _i}\cdot
\biggl[(n-l )^2-\sum^p_{j=1}(n_i-l _i)^2\biggr].\end{split}
\end{equation}
\begin{verbatim}
... \binom{n_i}{l _i}\\
\end{verbatim}
and
\begin{equation}\label{joe}
\begin{split}
H_c&=\frac12\sum^{n-1}_{l =0}
(-1)^{l}(n-l )^{p-2}
\sum_{l _1+\dots+l _p=l}
\prod^p_{i=1}\binom{n_i}{l _i}\\
&\quad\cdot[(n-l )-(n_i-l _i)]^{n_i-l _i}
\left(1-\frac{l _p}{n_p}\right)
[(n-l )-(n_p-l _p)].
\end{split}
\end{equation}

The enumeration of $H_c$ in a $K_{n_1\dotsm n_p}$ graph can also be
carried out by Theorem ~\ref{thm-H-param} or ~\ref{thm-asym}
together with the algebraic method of \eqref{multdef}.
Some elegant representations may be obtained. For example, $H_c$ in
a $K_{n_1n_2n_3}$ graph may be written
\begin{equation}\label{j:mark}
\begin{split}
H_c=&
\frac{n_1!\,n_2!\,n_3!}
{n_1+n_2+n_3}\sum_i\left[\binom{n_1}{i}
\binom{n_2}{n_3-n_1+i}\binom{n_3}{n_3-n_2+i}\right.\\
&+\left.\binom{n_1-1}{i}
\binom{n_2-1}{n_3-n_1+i}
\binom{n_3-1}{n_3-n_2+i}\right].\end{split}
\end{equation}

\section{Secret Key Exchanges}
\label{SKE}

Modern cryptography is fundamentally concerned with the problem of
secure private communication.  A Secret Key Exchange is a protocol
where Alice and Bob, having no secret information in common to start,
are able to agree on a common secret key, conversing over a public
channel.  The notion of a Secret Key Exchange protocol was first
introduced in the seminal paper of Diffie and Hellman
\cite{dihe:newdir}. \cite{dihe:newdir} presented a concrete
implementation of a Secret Key Exchange protocol, dependent on a
specific assumption (a variant on the discrete log), specially
tailored to yield Secret Key Exchange. Secret Key Exchange is of
course trivial if trapdoor permutations exist. However, there is no
known implementation based on a weaker general assumption.

The concept of an informationally one-way function was introduced
in \cite{imlelu:oneway}. We give only an informal definition here:

\begin{defn} A polynomial time
computable function $f = \{f_k\}$ is informationally
one-way if there is no probabilistic polynomial time algorithm which
(with probability of the form $1 - k^{-e}$ for some $e > 0$)
returns on input $y \in \{0,1\}^{k}$ a random element of $f^{-1}(y)$.
\end{defn}
In the non-uniform setting \cite{imlelu:oneway} show that these are not
weaker than one-way functions:
\begin{thm}[\cite{imlelu:oneway} (non-uniform)]
\label{th-info-ow-ow}
The existence of informationally one-way functions
implies the existence of one-way functions.
\end{thm}
We will stick to the convention introduced above of saying
``non-uniform'' before the theorem statement when the theorem
makes use of non-uniformity. It should be understood that
if nothing is said then the result holds for both the uniform and
the non-uniform models.

It now follows from \thmref{th-info-ow-ow} that

\begin{thm}[non-uniform]\label{th-weak-ske-owf} Weak SKE
implies the existence of a one-way function.
\end{thm}

More recently, the polynomial-time, interior point algorithms for linear
programming have been extended to the case of convex quadratic programs
\cite{moad:quadpro,ye:intalg}, certain linear complementarity problems
\cite{komiyo:lincomp,miyoki:lincomp}, and the nonlinear complementarity
problem \cite{komiyo:unipfunc}. The connection between these algorithms
and the classical Newton method for nonlinear equations is well
explained in \cite{komiyo:lincomp}.

\section{Review}
\label{computation}

We begin our discussion with the following definition:

\begin{defn}

A function $H\colon \Re^n \to \Re^n$ is said to be
\emph{B-differentiable} at the point $z$ if (i)~$H$ is Lipschitz
continuous in a neighborhood of $z$, and (ii)~ there exists a positive
homogeneous function $BH(z)\colon \Re^n \to \Re^n$, called the
\emph{B-derivative} of $H$ at $z$, such that
\[ \lim_{v \to 0} \frac{H(z+v) - H(z) - BH(z)v}{\enVert{v}} = 0. \]
The function $H$ is \textit{B-differentiable in set $S$} if it is
B-differentiable at every point in $S$. The B-derivative $BH(z)$ is said
to be \textit{strong} if
\[ \lim_{(v,v') \to (0,0)} \frac{H(z+v) - H(z+v') - BH(z)(v
 -v')}{\enVert{v - v'}} = 0. \]
\end{defn}


\begin{lem}\label{limbog} There exists a smooth function $\psi_0(z)$
defined for $\abs{z}>1-2a$ satisfying the following properties\textup{:}
\begin{enumerate}
\renewcommand{\labelenumi}{(\roman{enumi})}
\item $\psi_0(z)$ is bounded above and below by positive constants
$c_1\leq \psi_0(z)\leq c_2$.
\item If $\abs{z}>1$, then $\psi_0(z)=1$.
\item For all $z$ in the domain of $\psi_0$, $\Delta_0\ln \psi_0\geq 0$.
\item If $1-2a<\abs{z}<1-a$, then $\Delta_0\ln \psi_0\geq
c_3>0$.
\end{enumerate}
\end{lem}

\begin{proof}
We choose $\psi_0(z)$ to be a radial function depending only on $r=\abs{z}$.
Let $h(r)\geq 0$ be a suitable smooth function satisfying $h(r)\geq c_3$
for $1-2a<\abs{z}<1-a$, and $h(r)=0$ for $\abs{z}>1-\tfrac a2$. The radial
Laplacian
\[\Delta_0\ln\psi_0(r)=\left(\frac {d^2}{dr^2}+\frac
1r\frac d{dr}\right)\ln\psi_0(r)\]
has smooth coefficients for $r>1-2a$. Therefore, we may
apply the existence and uniqueness theory for ordinary differential
equations. Simply let $\ln \psi_0(r)$ be the solution of the differential
equation
\[\left(\frac{d^2}{dr^2}+\frac 1r\frac d{dr}\right)\ln \psi_0(r)=h(r)\]
with initial conditions given by $\ln \psi_0(1)=0$ and
$\ln\psi_0'(1)=0$.

Next, let $D_\nu$ be a finite collection of pairwise disjoint disks,
all of which are contained in the unit disk centered at the origin in
$C$. We assume that $D_\nu=\{z\mid \abs{z-z_\nu}<\delta\}$. Suppose that
$D_\nu(a)$ denotes the smaller concentric disk $D_\nu(a)=\{z\mid
\abs{z-z_\nu}\leq (1-2a)\delta\}$. We define a smooth weight function
$\Phi_0(z)$ for $z\in C-\bigcup_\nu D_\nu(a)$ by setting $\Phi_
0(z)=1$ when $z\notin \bigcup_\nu D_\nu$ and $\Phi_
0(z)=\psi_0((z-z_\nu)/\delta)$ when $z$ is an element of $D_\nu$. It
follows from \lemref{limbog} that $\Phi_ 0$ satisfies the properties:
\begin{enumerate}
\renewcommand{\labelenumi}{(\roman{enumi})}
\item \label{boundab}$\Phi_ 0(z)$ is bounded above and below by
positive constants $c_1\leq \Phi_ 0(z)\leq c_2$.
\item \label{d:over}$\Delta_0\ln\Phi_ 0\geq 0$ for all
$z\in C-\bigcup_\nu D_\nu(a)$,
the domain where the function $\Phi_ 0$ is defined.
\item \label{d:ad}$\Delta_0\ln\Phi_ 0\geq c_3\delta^{-2}$
when $(1-2a)\delta<\abs{z-z_\nu}<(1-a)\delta$.
\end{enumerate}
Let $A_\nu$ denote the annulus $A_\nu=\{(1-2a)\delta<\abs{z-z_\nu}<(1-a)
\delta \}$, and set $A=\bigcup_\nu A_\nu$. The
properties (\ref{d:over}) and (\ref{d:ad}) of $\Phi_ 0$
may be summarized as $\Delta_0\ln \Phi_ 0\geq c_3\delta^{-2}\chi_A$,
where $\chi _A$ is the characteristic function of $A$.
\end{proof}

Suppose that $\alpha$ is a nonnegative real constant. We apply
Proposition~\ref{prop:eg} with $\Phi(z)=\Phi_ 0(z) e^{\alpha\abs{z}^2}$. If
$u\in C^\infty_0(R^2-\bigcup_\nu D_\nu(a))$, assume that $\mathcal{D}$
is a bounded domain containing the support of $u$ and $A\subset
\mathcal{D}\subset R^2-\bigcup_\nu D_\nu(a)$. A calculation gives
\[\int_{\mathcal{D}}\abs{\overline\partial u}^2\Phi_ 0(z) e^{\alpha\abs{z}^2}
\geq c_4\alpha\int_{\mathcal{D}}\abs{u}^2\Phi_ 0e^{\alpha\abs{z}^2}
+c_5\delta^{-2}\int_ A\abs{u}^2\Phi_ 0e^{\alpha\abs{z}^2}.\]

The boundedness, property (\ref{boundab}) of $\Phi_ 0$, then yields
\[\int_{\mathcal{D}}\abs{\overline\partial u}^2e^{\alpha\abs{z}^2}\geq c_6\alpha
\int_{\mathcal{D}}\abs{u}^2e^{\alpha\abs{z}^2}
+c_7\delta^{-2}\int_ A\abs{u}^2e^{\alpha\abs{z}^2}.\]

Let $B(X)$ be the set of blocks of $\Lambda_{X}$
and let $b(X) = \abs{B(X)}$. If $\phi \in Q_{X}$ then
$\phi$ is constant on the blocks of $\Lambda_{X}$.
\begin{equation}\label{far-d}
 P_{X} = \{ \phi \in M \mid \Lambda_{\phi} = \Lambda_{X} \},
\qquad
Q_{X} = \{\phi \in M \mid \Lambda_{\phi} \geq \Lambda_{X} \}.
\end{equation}
If $\Lambda_{\phi} \geq \Lambda_{X}$ then
$\Lambda_{\phi} = \Lambda_{Y}$ for some $Y \geq X$ so that
\[ Q_{X} = \bigcup_{Y \geq X} P_{Y}. \]
Thus by M\"obius inversion
\[ \abs{P_{Y}}= \sum_{X\geq Y} \mu (Y,X)\abs{Q_{X}}.\]
Thus there is a bijection from $Q_{X}$ to $W^{B(X)}$.
In particular $\abs{Q_{X}} = w^{b(X)}$.

Next note that $b(X)=\dim X$. We see this by choosing a
basis for $X$ consisting of vectors $v^{k}$ defined by
\[v^{k}_{i}=
\begin{cases} 1 & \text{if $i \in \Lambda_{k}$},\\
0 &\text{otherwise.} \end{cases}
\]
\begin{verbatim}
\[v^{k}_{i}=
\begin{cases} 1 & \text{if $i \in \Lambda_{k}$},\\
0 &\text{otherwise.} \end{cases}
\]
\end{verbatim}

\begin{lem}\label{p0201}
Let $\A$ be an arrangement. Then
\[ \chi (\A,t) = \sum_{\B \subseteq \A}
(-1)^{\abs{\B}} t^{\dim T(\B)}. \]
\end{lem}

In order to compute $R''$ recall the definition
of $S(X,Y)$ from \lemref{lem-per}. Since $H \in \B$,
$\A_{H} \subseteq \B$. Thus if $T(\B) = Y$ then
$\B \in S(H,Y)$. Let $L'' = L(\A'')$. Then
\begin{equation}\label{E_SXgYy}
\begin{split}
R''&= \sum_{H\in \B \subseteq \A} (-1)^{\abs{\B}}
t^{\dim T(\B)}\\
&= \sum_{Y \in L''} \sum_{\B \in S(H,Y)}
(-1)^{\abs{\B}}t^{\dim Y} \\
&= -\sum_{Y \in L''} \sum_{\B \in S(H,Y)} (-1)^
{\abs{\B - \A_{H}}} t^{\dim Y} \\
&= -\sum_{Y \in L''} \mu (H,Y)t^{\dim Y} \\
&= -\chi (\A '',t).
\end{split}
\end{equation}

\begin{cor}\label{tripleA}
Let $(\A,\A',\A'')$ be a triple of arrangements. Then
\[ \pi (\A,t) = \pi (\A',t) + t \pi (\A'',t). \]
\end{cor}

\begin{defn}
Let $(\A,\A',\A'')$ be a triple with respect to
the hyperplane $H \in \A$. Call $H$ a \textit{separator}
if $T(\A) \not\in L(\A')$.
\end{defn}

\begin{cor}\label{nsep}
Let $(\A,\A',\A'')$ be a triple with respect to $H \in \A$.
\begin{enumerate}
\renewcommand{\labelenumi}{(\roman{enumi})}
\item
If $H$ is a separator then
\[ \mu (\A) = - \mu (\A'') \]
and hence
\[ \abs{\mu (\A)} = \abs{ \mu (\A'')}. \]

\item If $H$ is not a separator then
\[\mu (\A) = \mu (\A') - \mu (\A'') \]
and
\[ \abs{\mu (\A)} = \abs{\mu (\A')} + \abs{\mu (\A'')}. \]
\end{enumerate}
\end{cor}

\begin{proof}
It follows from \thmref{th-info-ow-ow} that $\pi(\A,t)$
has leading term
\[(-1)^{r(\A)}\mu (\A)t^{r(\A)}.\]
The conclusion
follows by comparing coefficients of the leading
terms on both sides of the equation in
Corollary~\ref{tripleA}. If $H$ is a separator then
$r(\A') < r(\A)$ and there is no contribution
from $\pi (\A',t)$.
\end{proof}

The Poincar\'e polynomial of an arrangement
will appear repeatedly
in these notes. It will be shown to equal the
Poincar\'e polynomial
of the graded algebras which we are going to
associate with $\A$. It is also the Poincar\'e
polynomial of the complement $M(\A)$ for a
complex arrangement. Here we prove
that the Poincar\'e polynomial is the chamber
counting function for a real arrangement. The
complement $M(\A)$ is a disjoint union of chambers
\[M(\A) = \bigcup_{C \in \Cham(\A)} C.\]
The number
of chambers is determined by the Poincar\'e
polynomial as follows.

\begin{thm}\label{th-realarr}
Let $\A_{\mathbf{R}}$ be a real arrangement. Then
\[ \abs{\Cham(\A_{\mathbf{R}})} = \pi (\A_{\mathbf{R}},1). \]
\end{thm}

\begin{proof}
We check the properties required in Corollary~\ref{nsep}:
(i) follows from $\pi (\Phi_{ l},t) = 1$, and (ii) is a
consequence of Corollary~\ref{BI}.
\end{proof}

\begin{figure}
\vspace{5cm}
\caption[]{$Q(\A_{1}) = xyz(x-z)(x+z)(y-z)(y+z)$}
\end{figure}

\begin{figure}
\vspace{5cm}
\caption[]{$Q(\A_{2})= xyz(x+y+z)(x+y-z)(x-y+z)(x-y-z)$}
\end{figure}


\begin{thm}
\label{T_first_the_int}
Let $\phi$ be a protocol for a random pair $\XcY$.
If one of $\st_\phi(x',y)$ and $\st_\phi(x,y')$ is a prefix of the other
and $(x,y)\in\SXY$, then
\[
\langle \st_j(x',y)\rangle_{j=1}^\infty
=\langle \st_j(x,y)\rangle_{j=1}^\infty
=\langle \st_j(x,y')\rangle_{j=1}^\infty .
\]
\end{thm}
\begin{proof}
We show by induction on $i$ that
\[
\langle \st_j(x',y)\rangle_{j=1}^i
=\langle \st_j(x,y)\rangle_{j=1}^i
=\langle \st_j(x,y')\rangle_{j=1}^i.
\]
The induction hypothesis holds vacuously for $i=0$. Assume it holds for
$i-1$, in particular
$[\st_j(x',y)]_{j=1}^{i-1}=[\st_j(x,y')]_{j=1}^{i-1}$. Then one of
$[\st_j(x',y)]_{j=i}^{\infty}$ and $[\st_j(x,y')]_{j=i}^{\infty}$ is a
prefix of the other which implies that one of $\st_i(x',y)$ and
$\st_i(x,y')$ is a prefix of the other. If the $i$th message is
transmitted by $P_\X$ then, by the separate-transmissions property and
the induction hypothesis, $\st_i(x,y)=\st_i(x,y')$, hence one of
$\st_i(x,y)$ and $\st_i(x',y)$ is a prefix of the other. By the
implicit-termination property, neither $\st_i(x,y)$ nor $\st_i(x',y)$
can be a proper prefix of the other, hence they must be the same and
$\st_i(x',y)=\st_i(x,y)=\st_i(x,y')$. If the $i$th message is
transmitted by $\PY$ then, symmetrically, $\st_i(x,y)=\st_i(x',y)$ by
the induction hypothesis and the separate-transmissions property, and,
then, $\st_i(x,y)=\st_i(x,y')$ by the implicit-termination property,
proving the induction step.
\end{proof}

If $\phi$ is a protocol for $(X,Y)$, and $(x,y)$, $(x',y)$ are distinct
inputs in $\SXY$, then, by the correct-decision property,
$\langle\st_j(x,y)\rangle_{j=1}^\infty\ne\langle
\st_j(x',y)\rangle_{j=1}^\infty$.

Equation~(\ref{E_SXgYy}) defined $\PY$'s ambiguity set $\SXgYy$
to be the set of possible $X$ values when $Y=y$.
The last corollary implies that for all $y\in\SY$,
the multiset%
\footnote{A multiset allows multiplicity of elements.
Hence, $\{0,01,01\}$ is prefix free as a set, but not as a multiset.}
of codewords $\{\st_\phi(x,y):x\in\SXgYy\}$ is prefix free.

\section{One-Way Complexity}
\label{S_Cp1}

$\Cw1$, the one-way complexity of a random pair $\XcY$,
is the number of bits $P_\X$ must transmit in the worst case
when $\PY$ is not permitted to transmit any feedback messages.
Starting with $\SXY$, the support set of $\XcY$, we define $\G$,
the \textit{characteristic hypergraph} of $\XcY$, and show that
\[
\Cw1=\lceil\,\log\chi(\G)\rceil\ .
\]

Let $\XcY$ be a random pair. For each $y$ in $\SY$, the support set of
$Y$, Equation~(\ref{E_SXgYy}) defined $\SXgYy$ to be the set of possible
$x$ values when $Y=y$. The \textit{characteristic hypergraph} $\G$ of
$\XcY$ has $\SX$ as its vertex set and the hyperedge $\SXgYy$ for each
$y\in\SY$.


We can now prove a continuity theorem.
\begin{thm}\label{t:conl}
Let $\Omega \subset\mathbf{R}^n$ be an open set, let
$u\in BV(\Omega ;\mathbf{R}^m)$, and let
\begin{equation}\label{quts}
T^u_x=\left\{y\in\mathbf{R}^m:
 y=\tilde u(x)+\left\langle \frac{Du}{\abs{Du}}(x),z
\right\rangle \text{ for some }z\in\mathbf{R}^n\right\}
\end{equation}
for every $x\in\Omega \backslash S_u$. Let $f\colon \mathbf{R}^m\to
\mathbf{R}^k$ be a Lipschitz continuous function such that $f(0)=0$, and
let $v=f(u)\colon \Omega \to \mathbf{R}^k$. Then $v\in BV(\Omega
;\mathbf{R}^k)$ and
\begin{equation}
Jv=\eval{(f(u^+)-f(u^-))\otimes \nu_u\cdot\,
\mathcal{H}_{n-1}}_{S_u}.
\end{equation}
In addition, for $\abs{\wt{D}u}$-almost every $x\in\Omega $ the
restriction of the function $f$ to $T^u_x$ is differentiable at $\tilde
u(x)$ and
\begin{equation}
\wt{D}v=\nabla (\eval{f}_{T^u_x})(\tilde u)
\frac{\wt{D}u}{\abs{\wt{D}u}}\cdot\abs{\wt{D}u}.\end{equation}
\end{thm}

Before proving the theorem, we state without proof three elementary
remarks which will be useful in the sequel.
\begin{rem}\label{r:omb}
Let $\omega\colon \left]0,+\infty\right[\to \left]0,+\infty\right[$
be a continuous function such that $\omega (t)\to 0$ as $t\to
0$. Then
\[\lim_{h\to 0^+}g(\omega(h))=L\Leftrightarrow\lim_{h\to
0^+}g(h)=L\]
for any function $g\colon \left]0,+\infty\right[\to \mathbf{R}$.
\end{rem}
\begin{rem}\label{r:dif}
Let $g \colon  \mathbf{R}^n\to \mathbf{R}$ be a Lipschitz
continuous function and assume that
\[L(z)=\lim_{h\to 0^+}\frac{g(hz)-g(0)}h\]
exists for every $z\in\mathbf{Q}^n$ and that $L$ is a linear function of
$z$. Then $g$ is differentiable at 0.
\end{rem}
\begin{rem}\label{r:dif0}
Let $A \colon \mathbf{R}^n\to \mathbf{R}^m$ be a linear function, and
let $f \colon \mathbf{R}^m\to \mathbf{R}$ be a function. Then the
restriction of $f$ to the range of $A$ is differentiable at 0 if and
only if $f(A)\colon \mathbf{R}^n\to \mathbf{R}$ is differentiable at 0
and
\[\nabla(\eval{f}_{\IM(A)})(0)A=\nabla (f(A))(0).\]
\end{rem}

\begin{proof}
 We begin by showing that $v\in BV(\Omega;\mathbf{R}^k)$ and
\begin{equation}\label{e:bomb}
\abs{Dv}(B)\le K\abs{Du}(B)\qquad\forall B\in\mathbf{B}(\Omega ),
\end{equation}
where $K>0$ is the Lipschitz constant of $f$. By \eqref{sum-Di} and by
the approximation result quoted in \secref{s:mt}, it is possible to find
a sequence $(u_h)\subset C^1(\Omega ;\mathbf{R}^m)$ converging to $u$ in
$L^1(\Omega ;\mathbf{R}^m)$ and such that
\[\lim_{h\to +\infty}\int_\Omega \abs{\nabla u_h}\,dx=\abs{Du}(\Omega ).\]
The functions $v_h=f(u_h)$ are locally Lipschitz continuous in $\Omega
$, and the definition of differential implies that $\abs{\nabla v_h}\le
K\abs{\nabla u_h}$ almost everywhere in $\Omega $. The lower semicontinuity
of the total variation and \eqref{sum-Di} yield
\begin{equation}
\begin{split}
\abs{Dv}(\Omega )\le\liminf_{h\to +\infty}\abs{Dv_h}(\Omega) &
=\liminf_{h\to +\infty}\int_\Omega \abs{\nabla v_h}\,dx\\
&\le K\liminf_{h\to +\infty}\int_\Omega
\abs{\nabla u_h}\,dx=K\abs{Du}(\Omega).
\end{split}\end{equation}
Since $f(0)=0$, we have also
\[\int_\Omega \abs{v}\,dx\le K\int_\Omega \abs{u}\,dx;\]
therefore $u\in BV(\Omega ;\mathbf{R}^k)$. Repeating the same argument
for every open set $A\subset\Omega $, we get \eqref{e:bomb} for every
$B\in\mathbf{B}(\Omega)$, because $\abs{Dv}$, $\abs{Du}$ are Radon measures. To
prove \lemref{limbog}, first we observe that
\begin{equation}\label{e:SS}
S_v\subset S_u,\qquad\tilde v(x)=f(\tilde u(x))\qquad \forall x\in\Omega
\backslash S_u.\end{equation}
In fact, for every $\varepsilon >0$ we have
\[\{y\in B_\rho(x): \abs{v(y)-f(\tilde u(x))}>\varepsilon \}\subset \{y\in
B_\rho(x): \abs{u(y)-\tilde u(x)}>\varepsilon /K\},\]
hence
\[\lim_{\rho\to 0^+}\frac{\abs{\{y\in B_\rho(x): \abs{v(y)-f(\tilde u(x))}>
\varepsilon \}}}{\rho^n}=0\]
whenever $x\in\Omega \backslash S_u$. By a similar argument, if $x\in
S_u$ is a point such that there exists a triplet $(u^+,u^-,\nu_u)$
satisfying \eqref{detK1}, \eqref{detK2}, then
\[
(v^+(x)-v^-(x))\otimes \nu_v=(f(u^+(x))-f(u^-(x)))\otimes\nu_u\quad
\text{if }x\in S_v
\]
and $f(u^-(x))=f(u^+(x))$ if $x\in S_u\backslash S_v$. Hence, by (1.8)
we get
\begin{equation*}\begin{split}
Jv(B)=\int_{B\cap S_v}(v^+-v^-)\otimes \nu_v\,d\mathcal{H}_{n-1}&=
\int_{B\cap S_v}(f(u^+)-f(u^-))\otimes \nu_u\,d\mathcal{H}_{n-1}\\
&=\int_{B\cap S_u}(f(u^+)-f(u^-))\otimes \nu_u\,d\mathcal{H}_{n-1}
\end{split}\end{equation*}
and \lemref{limbog} is proved.
\end{proof}

To prove \eqref{e:SS}, it is not restrictive to assume that $k=1$.
Moreover, to simplify our notation, from now on we shall assume that
$\Omega = \mathbf{R}^n$. The proof of \eqref{e:SS} is divided into two
steps. In the first step we prove the statement in the one-dimensional
case $(n=1)$, using \thmref{th-weak-ske-owf}. In the second step we
achieve the general result using \thmref{t:conl}.

\subsection*{Step 1}
Assume that $n=1$. Since $S_u$ is at most countable, \eqref{sum-bij}
yields that $\abs{\wt{D}v}(S_u\backslash S_v)=0$, so that
\eqref{e:st} and \eqref{e:barwq} imply that $Dv=\wt{D}v+Jv$ is
the Radon-Nikod\'ym decomposition of $Dv$ in absolutely continuous and
singular part with respect to $\abs{\wt{D} u}$. By
\thmref{th-weak-ske-owf}, we have
\begin{equation*}
\frac{\wt{D}v}{\abs{\wt{D}u}}(t)=\lim_{s\to t^+}
\frac{Dv(\interval{\left[t,s\right[})}
{\abs{\wt{D}u}(\interval{\left[t,s\right[})},\qquad
\frac{\wt{D}u}{\abs{\wt{D}u}}(t)=\lim_{s\to t^+}
\frac{Du(\interval{\left[t,s\right[})}
{\abs{\wt{D}u}(\interval{\left[t,s\right[})}
\end{equation*}
$\abs{\wt{D}u}$-almost everywhere in $\mathbf{R}$. It is well known
(see, for instance, \cite[2.5.16]{ste:sint}) that every one-dimensional
function of bounded variation $w$ has a unique left continuous
representative, i.e., a function $\hat w$ such that $\hat w=w$ almost
everywhere and $\lim_{s\to t^-}\hat w(s)=\hat w(t)$ for every $t\in
\mathbf{R}$. These conditions imply
\begin{equation}
\hat u(t)=Du(\interval{\left]-\infty,t\right[}),
\qquad \hat v(t)=Dv(\interval{\left]-\infty,t\right[})\qquad
\forall t\in\mathbf{R}
\end{equation}
and
\begin{equation}\label{alimo}
\hat v(t)=f(\hat u(t))\qquad\forall t\in\mathbf{R}.\end{equation}
Let $t\in\mathbf{R}$ be such that
$\abs{\wt{D}u}(\interval{\left[t,s\right[})>0$ for every $s>t$ and
assume that the limits in \eqref{joe} exist. By \eqref{j:mark} and
\eqref{far-d} we get
\begin{equation*}\begin{split}
\frac{\hat v(s)-\hat
v(t)}{\abs{\wt{D}u}(\interval{\left[t,s\right[})}&=\frac {f(\hat
u(s))-f(\hat u(t))}{\abs{\wt{D}u}(\interval{\left[t,s\right[})}\\
&=\frac{f(\hat u(s))-f(\hat
u(t)+\dfrac{\wt{D}u}{\abs{\wt{D}u}}(t)\abs{\wt{D}u
}(\interval{\left[t,s\right[}))}%
{\abs{\wt{D}u}(\interval{\left[t,s\right[})}\\
&+\frac
{f(\hat u(t)+\dfrac{\wt{D}u}{\abs{\wt{D}u}}(t)\abs{\wt{D}
u}(\interval{\left[t,s\right[}))-f(\hat
u(t))}{\abs{\wt{D}u}(\interval{\left[t,s\right[})}
\end{split}\end{equation*}
for every $s>t$. Using the Lipschitz condition on $f$ we find
{\setlength{\multlinegap}{0pt}
\begin{multline*}
\left\lvert\frac{\hat v(s)-\hat
v(t)}{\abs{\wt{D}u}(\interval{\left[t,s\right[})} -\frac{f(\hat
u(t)+\dfrac{\wt{D}u}{\abs{\wt{D}u}}(t)
\abs{\wt{D}u}(\interval{\left[t,s\right[}))-f(\hat
u(t))}{\abs{\wt{D}u}(\interval{\left[t,s\right[})}\right\rvert\\
\le K\left\lvert
\frac{\hat u(s)-\hat u(t)}
  {\abs{\wt{D}u}(\interval{\left[t,s\right[})}
-\frac{\wt{D}u}{\abs{
\wt{D}u}}(t)\right\rvert.\end{multline*}
}% end of group with \multlinegap=0pt
By \eqref{e:bomb}, the function $s\to
\abs{\wt{D}u}(\interval{\left[t,s\right[})$ is continuous and
converges to 0 as $s\downarrow t$. Therefore Remark~\ref{r:omb} and the
previous inequality imply
\[\frac{\wt{D}v}{\abs{\wt{D}u}}(t)=\lim_{h\to 0^+}
\frac{f(\hat u(t)+h\dfrac{\wt{D}u}{\abs{\wt{D}u}}
(t))-f(\hat u(t))}h\quad\abs{\wt{D}u}\text{-a.e. in }\mathbf{R}.\]
By \eqref{joe}, $\hat u(x)=\tilde u(x)$ for every
$x\in\mathbf{R}\backslash S_u$; moreover, applying the same argument to
the functions $u'(t)=u(-t)$, $v'(t)=f(u'(t))=v(-t)$, we get
\[\frac{\wt{D}v}{\abs{\wt{D}u}}(t)=\lim_{h\to 0}
\frac{f(\tilde u(t)
+h\dfrac{\wt{D}u}{\abs{\wt{D}u}}(t))-f(\tilde u(t))}{h}
\qquad\abs{\wt{D}u}\text{-a.e. in }\mathbf{R}\]
and our statement is proved.

\subsection*{Step 2}

Let us consider now the general case $n>1$. Let $\nu\in \mathbf{R}^n$ be
such that $\abs{\nu}=1$, and let $\pi_\nu=\{y\in\mathbf{R}^n: \langle
y,\nu\rangle =0\}$. In the following, we shall identify $\mathbf{R}^n$
with $\pi_\nu\times\mathbf{R}$, and we shall denote by $y$ the variable
ranging in $\pi_\nu$ and by $t$ the variable ranging in $\mathbf{R}$. By
the just proven one-dimensional result, and by \thmref{thm-main}, we get
\[\lim_{h\to 0}\frac{f(\tilde u(y+t\nu)+h\dfrac{\wt{D}u_y}{\abs{
\wt{D}u_y}}(t))-f(\tilde u(y+t\nu))}h=\frac{\wt{D}v_y}{\abs{
\wt{D}u_y}}(t)\qquad\abs{\wt{D}u_y}\text{-a.e. in }\mathbf{R}\]
for $\mathcal{H}_{n-1}$-almost every $y\in \pi_\nu$. We claim that
\begin{equation}
\frac{\langle \wt{D}u,\nu\rangle }{\abs{\langle \wt{D}u,\nu\rangle
}}(y+t\nu)=\frac{\wt{D}u_y}
{\abs{\wt{D}u_y}}(t)\qquad\abs{\wt{D}u_y}\text{-a.e. in }\mathbf{R}
\end{equation}
for $\mathcal{H}_{n-1}$-almost every $y\in\pi_\nu$. In fact, by
\eqref{sum-ali} and \eqref{delta-l} we get
\begin{multline*}
\int_{\pi_\nu}\frac{\wt{D}u_y}{\abs{\wt{D}u_y}}\cdot\abs{\wt{D}u_y
}\,d\mathcal{H}_{n-1}(y)=\int_{\pi_\nu}\wt{D}u_y\,d\mathcal{H}_{n-1}(y)\\
=\langle \wt{D}u,\nu\rangle =\frac
{\langle \wt{D}u,\nu\rangle }{\abs{\langle \wt{D}u,\nu\rangle}}\cdot
\abs{\langle \wt{D}u,\nu\rangle }=\int_{\pi_\nu}\frac{
\langle \wt{D}u,\nu\rangle }{\abs{\langle \wt{D}u,\nu\rangle }}
(y+\cdot \nu)\cdot\abs{\wt{D}u_y}\,d\mathcal{H}_{n-1}(y)
\end{multline*}
and \eqref{far-d} follows from \eqref{sum-Di}. By the same argument it
is possible to prove that
\begin{equation}
\frac{\langle \wt{D}v,\nu\rangle }{\abs{\langle \wt{D}u,\nu\rangle
}}(y+t\nu)=\frac{\wt{D}v_y}{\abs{\wt{D}u_y}}(t)\qquad\abs{
\wt{D}u_y}\text{-a.e. in }\mathbf{R}\end{equation}
for $\mathcal{H}_{n-1}$-almost every $y\in \pi_\nu$. By \eqref{far-d}
and \eqref{E_SXgYy} we get
\[
\lim_{h\to 0}\frac{f(\tilde u(y+t\nu)+h\dfrac{\langle \wt{D}
u,\nu\rangle }{\abs{\langle \wt{D}u,\nu\rangle }}(y+t\nu))-f(\tilde
u(y+t\nu))}{h}
=\frac{\langle \wt{D}v,\nu\rangle }{\abs{\langle
\wt{D}u,\nu\rangle }}(y+t\nu)\]
for $\mathcal{H}_{n-1}$-almost every $y\in\pi_\nu$, and using again
\eqref{detK1}, \eqref{detK2} we get
\[
\lim_{h\to 0}\frac{f(\tilde u(x)+h\dfrac{\langle
\wt{D}u,\nu\rangle }{\abs{\langle \wt{D}u,\nu\rangle }}(x))-f(\tilde
u(x))}{h}=\frac{\langle \wt{D}v,\nu\rangle }{\abs{\langle \wt{D}u,\nu
\rangle }}(x)
\]
$\abs{\langle \wt{D}u,\nu\rangle}$-a.e. in $\mathbf{R}^n$.

Since the function $\abs{\langle \wt{D}u,\nu\rangle }/\abs{\wt{D}u}$
is strictly positive $\abs{\langle \wt{D}u,\nu\rangle }$-almost everywhere,
we obtain also
\begin{multline*}
\lim_{h\to 0}\frac{f(\tilde u(x)+h\dfrac{\abs{\langle
\wt{D}u,\nu\rangle }}{\abs{\wt{D}u}}(x)\dfrac{\langle \wt{D}
u,\nu\rangle }{\abs{\langle \wt{D}u,\nu\rangle }}(x))-f(\tilde u(x))}{h}\\
=\frac{\abs{\langle \wt{D}u,\nu\rangle }}{\abs{\wt{D}u}}(x)\frac
{\langle \wt{D}v,\nu\rangle }{\abs{\langle
\wt{D}u,\nu\rangle }}(x)
\end{multline*}
$\abs{\langle \wt{D}u,\nu\rangle }$-almost everywhere in $\mathbf{R}^n$.

Finally, since
\begin{align*}
&\frac{\abs{\langle \wt{D}u,\nu\rangle }}{\abs{\wt{D}u}}
\frac{\langle \wt{D}u,\nu\rangle }{\abs{\langle \wt{D}u,\nu\rangle}}
=\frac{\langle \wt{D}u,\nu\rangle }{\abs{\wt{D}u}}
=\left\langle \frac{\wt{D}u}{\abs{\wt{D}u}},\nu\right\rangle
        \qquad\abs{\wt{D}u}\text{-a.e. in }\mathbf{R}^n\\
&\frac{\abs{\langle \wt{D}u,\nu\rangle }}{\abs{\wt{D}u}}
\frac{\langle \wt{D}v,\nu\rangle }{\abs{\langle \wt{D}u,\nu\rangle}}
=\frac{\langle \wt{D}v,\nu\rangle }{\abs{\wt{D}u}}
=\left\langle \frac{\wt{D}v}{\abs{\wt{D}u}},\nu\right\rangle
        \qquad\abs{\wt{D}u}\text{-a.e. in }\mathbf{R}^n
\end{align*}
and since both sides of \eqref{alimo}
are zero $\abs{\wt{D}u}$-almost everywhere
on $\abs{\langle \wt{D}u,\nu\rangle }$-negligible sets, we conclude that
\[
\lim_{h\to 0}\frac{f\left(
\tilde u(x)+h\left\langle \dfrac{\wt{D}
u}{\abs{\wt{D}u}}(x),\nu\right\rangle \right)-f(\tilde u(x))}h
=\left\langle \frac{\wt{D}v}{\abs{\wt{D}u}}(x),\nu\right\rangle,
\]
$\abs{\wt{D}u}$-a.e. in $\mathbf{R}^n$.
Since $\nu$ is arbitrary, by Remarks \ref{r:dif} and~\ref{r:dif0}
the restriction of $f$ to
the affine space $T^u_x$ is differentiable at $\tilde u(x)$ for $\abs{\wt{D}
u}$-almost every $x\in \mathbf{R}^n$ and \eqref{quts} holds.\qed

It follows from \eqref{sum-Di}, \eqref{detK1}, and \eqref{detK2} that
\begin{equation}\label{Dt}
D(t_1,\dots,t_n)=\sum_{I\in\mathbf{n}}(-1)^{\abs{I}-1}\abs{I}
\prod_{i\in I}t_i\prod_{j\in I}(D_j+\lambda_jt_j)\det\mathbf{A}^{(\lambda)}
(\overline I|\overline I).
\end{equation}
Let $t_i=\hat x_i$, $i=1,\dots,n$. Lemma 1 leads to
\begin{equation}\label{Dx}
D(\hat x_1,\dots,\hat x_n)=\prod_{i\in\mathbf{n}}\hat x_i
\sum_{I\in\mathbf{n}}(-1)^{\abs{I}-1}\abs{I}\per \mathbf{A}
^{(\lambda)}(I|I)\det\mathbf{A}^{(\lambda)}(\overline I|\overline I).
\end{equation}
By \eqref{H-cycles}, \eqref{sum-Di}, and \eqref{Dx},
we have the following result:
\begin{thm}\label{thm-H-param}
\begin{equation}\label{H-param}
H_c=\frac{1}{2n}\sum^n_{l =1}l (-1)^{l -1}A_{l}
^{(\lambda)},
\end{equation}
where
\begin{equation}\label{A-l-lambda}
A^{(\lambda)}_l =\sum_{I_l \subseteq\mathbf{n}}\per \mathbf{A}
^{(\lambda)}(I_l |I_l )\det\mathbf{A}^{(\lambda)}
(\overline I_{l}|\overline I_l ),\abs{I_{l}}=l .
\end{equation}
\end{thm}

It is worth noting that $A_l ^{(\lambda)}$ of \eqref{A-l-lambda} is
similar to the coefficients $b_l $ of the characteristic polynomial of
\eqref{bl-sum}. It is well known in graph theory that the coefficients
$b_l $ can be expressed as a sum over certain subgraphs. It is
interesting to see whether $A_l $, $\lambda=0$, structural properties
of a graph.

We may call \eqref{H-param} a parametric representation of $H_c$. In
computation, the parameter $\lambda_i$ plays very important roles. The
choice of the parameter usually depends on the properties of the given
graph. For a complete graph $K_n$, let $\lambda_i=1$, $i=1,\dots,n$.
It follows from \eqref{A-l-lambda} that
\begin{equation}\label{compl-gr}
A^{(1)}_l =\begin{cases} n!,&\text{if }l =1\\
0,&\text{otherwise}.\end{cases}
\end{equation}
By \eqref{H-param}
\begin{equation}
H_c=\frac 12(n-1)!.
\end{equation}
For a complete bipartite graph $K_{n_1n_2}$, let $\lambda_i=0$, $i=1,\dots,n$.
By \eqref{A-l-lambda},
\begin{equation}
A_l =
\begin{cases} -n_1!n_2!\delta_{n_1n_2},&\text{if }l =2\\
0,&\text{otherwise }.\end{cases}
\label{compl-bip-gr}
\end{equation}
Theorem ~\ref{thm-H-param}
leads to
\begin{equation}
H_c=\frac1{n_1+n_2}n_1!n_2!\delta_{n_1n_2}.
\end{equation}

Now, we consider an asymmetrical approach. Theorem \ref{thm-main} leads to
\begin{multline}
\det\mathbf{K}(t=1,t_1,\dots,t_n;l |l )\\
=\sum_{I\subseteq\mathbf{n}-\{l \}}
(-1)^{\abs{I}}\prod_{i\in I}t_i\prod_{j\in I}
(D_j+\lambda_jt_j)\det\mathbf{A}^{(\lambda)}
(\overline I\cup\{l \}|\overline I\cup\{l \}).
\end{multline}

By \eqref{H-cycles} and \eqref{sum-ali} we have the following asymmetrical
result:
\begin{thm}\label{thm-asym}
\begin{equation}
H_c=\frac12\sum_{I\subseteq\mathbf{n}-\{l \}}
(-1)^{\abs{I}}\per\mathbf{A}^{(\lambda)}(I|I)\det
\mathbf{A}^{(\lambda)}
(\overline I\cup\{l \}|\overline I\cup\{l \})
\end{equation}
which reduces to Goulden--Jackson's formula when $\lambda_i=0,i=1,\dots,n$
\cite{mami:matrixth}.
\end{thm}

\section{Various font features of the \pkg{amsmath} package}
\label{s:font}
\subsection{Bold versions of special symbols}

In the \pkg{amsmath} package \cn{boldsymbol} is used for getting
individual bold math symbols and bold Greek letters---everything in
math except for letters of the Latin alphabet,
where you'd use \cn{mathbf}.  For example,
\begin{verbatim}
A_\infty + \pi A_0 \sim
\mathbf{A}_{\boldsymbol{\infty}} \boldsymbol{+}
\boldsymbol{\pi} \mathbf{A}_{\boldsymbol{0}}
\end{verbatim}
looks like this:
\[A_\infty + \pi A_0 \sim \mathbf{A}_{\boldsymbol{\infty}}
\boldsymbol{+} \boldsymbol{\pi} \mathbf{A}_{\boldsymbol{0}}\]

\subsection{``Poor man's bold''}
If a bold version of a particular symbol doesn't exist in the
available fonts,
then \cn{boldsymbol} can't be used to make that symbol bold.
At the present time, this means that
\cn{boldsymbol} can't be used with symbols from
the \fn{msam} and \fn{msbm} fonts, among others.
In some cases, poor man's bold (\cn{pmb}) can be used instead
of \cn{boldsymbol}:
%  Can't show example from msam or msbm because this document is
%  supposed to be TeXable even if the user doesn't have
%  AMSFonts.  MJD 5-JUL-1990
\[\frac{\partial x}{\partial y}
\pmb{\bigg\vert}
\frac{\partial y}{\partial z}\]
\begin{verbatim}
\[\frac{\partial x}{\partial y}
\pmb{\bigg\vert}
\frac{\partial y}{\partial z}\]
\end{verbatim}
So-called ``large operator'' symbols such as $\sum$ and $\prod$
require an additional command, \cn{mathop},
to produce proper spacing and limits when \cn{pmb} is used.
For further details see \textit{The \TeX book}.
\[\sum_{\substack{i<B\\\text{$i$ odd}}}
\prod_\kappa \kappa F(r_i)\qquad
\mathop{\pmb{\sum}}_{\substack{i<B\\\text{$i$ odd}}}
\mathop{\pmb{\prod}}_\kappa \kappa(r_i)
\]
\begin{verbatim}
\[\sum_{\substack{i<B\\\text{$i$ odd}}}
\prod_\kappa \kappa F(r_i)\qquad
\mathop{\pmb{\sum}}_{\substack{i<B\\\text{$i$ odd}}}
\mathop{\pmb{\prod}}_\kappa \kappa(r_i)
\]
\end{verbatim}

\section{Compound symbols and other features}
\label{s:comp}
\subsection{Multiple integral signs}

\cn{iint}, \cn{iiint}, and \cn{iiiint} give multiple integral signs
with the spacing between them nicely adjusted,  in both text and
display style.  \cn{idotsint} gives two integral signs with dots
between them.
\begin{gather}
\iint\limits_A f(x,y)\,dx\,dy\qquad\iiint\limits_A
f(x,y,z)\,dx\,dy\,dz\\
\iiiint\limits_A
f(w,x,y,z)\,dw\,dx\,dy\,dz\qquad\idotsint\limits_A f(x_1,\dots,x_k)
\end{gather}

\subsection{Over and under arrows}

Some extra over and under arrow operations are provided in
the \pkg{amsmath} package.  (Basic \LaTeX\ provides
\cn{overrightarrow} and \cn{overleftarrow}).
\begin{align*}
\overrightarrow{\psi_\delta(t) E_t h}&
=\underrightarrow{\psi_\delta(t) E_t h}\\
\overleftarrow{\psi_\delta(t) E_t h}&
=\underleftarrow{\psi_\delta(t) E_t h}\\
\overleftrightarrow{\psi_\delta(t) E_t h}&
=\underleftrightarrow{\psi_\delta(t) E_t h}
\end{align*}
\begin{verbatim}
\begin{align*}
\overrightarrow{\psi_\delta(t) E_t h}&
=\underrightarrow{\psi_\delta(t) E_t h}\\
\overleftarrow{\psi_\delta(t) E_t h}&
=\underleftarrow{\psi_\delta(t) E_t h}\\
\overleftrightarrow{\psi_\delta(t) E_t h}&
=\underleftrightarrow{\psi_\delta(t) E_t h}
\end{align*}
\end{verbatim}
These all scale properly in subscript sizes:
\[\int_{\overrightarrow{AB}} ax\,dx\]
\begin{verbatim}
\[\int_{\overrightarrow{AB}} ax\,dx\]
\end{verbatim}

\subsection{Dots}

Normally you need only type \cn{dots} for ellipsis dots in a
math formula.  The main exception is when the dots
fall at the end of the formula; then you need to
specify one of \cn{dotsc} (series dots, after a comma),
\cn{dotsb} (binary dots, for binary relations or operators),
\cn{dotsm} (multiplication dots), or \cn{dotsi} (dots after
an integral).  For example, the input
\begin{verbatim}
Then we have the series $A_1,A_2,\dotsc$,
the regional sum $A_1+A_2+\dotsb$,
the orthogonal product $A_1A_2\dotsm$,
and the infinite integral
\[\int_{A_1}\int_{A_2}\dotsi\].
\end{verbatim}
produces
\begin{quotation}
Then we have the series $A_1,A_2,\dotsc$,
the regional sum $A_1+A_2+\dotsb$,
the orthogonal product $A_1A_2\dotsm$,
and the infinite integral
\[\int_{A_1}\int_{A_2}\dotsi\]
\end{quotation}

\subsection{Accents in math}

Double accents:
\[\Hat{\Hat{H}}\quad\Check{\Check{C}}\quad
\Tilde{\Tilde{T}}\quad\Acute{\Acute{A}}\quad
\Grave{\Grave{G}}\quad\Dot{\Dot{D}}\quad
\Ddot{\Ddot{D}}\quad\Breve{\Breve{B}}\quad
\Bar{\Bar{B}}\quad\Vec{\Vec{V}}\]
\begin{verbatim}
\[\Hat{\Hat{H}}\quad\Check{\Check{C}}\quad
\Tilde{\Tilde{T}}\quad\Acute{\Acute{A}}\quad
\Grave{\Grave{G}}\quad\Dot{\Dot{D}}\quad
\Ddot{\Ddot{D}}\quad\Breve{\Breve{B}}\quad
\Bar{\Bar{B}}\quad\Vec{\Vec{V}}\]
\end{verbatim}
This double accent operation is complicated
and tends to slow down the processing of a \LaTeX\ file.


\subsection{Dot accents}
\cn{dddot} and \cn{ddddot} are available to
produce triple and quadruple dot accents
in addition to the \cn{dot} and \cn{ddot} accents already available
in \LaTeX:
\[\dddot{Q}\qquad\ddddot{R}\]
\begin{verbatim}
\[\dddot{Q}\qquad\ddddot{R}\]
\end{verbatim}

\subsection{Roots}

In the \pkg{amsmath} package \cn{leftroot} and \cn{uproot} allow you to adjust
the position of the root index of a radical:
\begin{verbatim}
\sqrt[\leftroot{-2}\uproot{2}\beta]{k}
\end{verbatim}
gives good positioning of the $\beta$:
\[\sqrt[\leftroot{-2}\uproot{2}\beta]{k}\]

\subsection{Boxed formulas} The command \cn{boxed} puts a box around its
argument, like \cn{fbox} except that the contents are in math mode:
\begin{verbatim}
\boxed{W_t-F\subseteq V(P_i)\subseteq W_t}
\end{verbatim}
\[\boxed{W_t-F\subseteq V(P_i)\subseteq W_t}.\]

\subsection{Extensible arrows}
\cn{xleftarrow} and \cn{xrightarrow} produce
arrows that extend automatically to accommodate unusually wide
subscripts or superscripts.  The text of the subscript or superscript
are given as an optional resp.\@ mandatory argument:
Example:
\[0 \xleftarrow[\zeta]{\alpha} F\times\triangle[n-1]
  \xrightarrow{\partial_0\alpha(b)} E^{\partial_0b}\]
\begin{verbatim}
\[0 \xleftarrow[\zeta]{\alpha} F\times\triangle[n-1]
  \xrightarrow{\partial_0\alpha(b)} E^{\partial_0b}\]
\end{verbatim}

\subsection{\cn{overset}, \cn{underset}, and \cn{sideset}}
Examples:
\[\overset{*}{X}\qquad\underset{*}{X}\qquad
\overset{a}{\underset{b}{X}}\]
\begin{verbatim}
\[\overset{*}{X}\qquad\underset{*}{X}\qquad
\overset{a}{\underset{b}{X}}\]
\end{verbatim}

The command \cn{sideset} is for a rather special
purpose: putting symbols at the subscript and superscript
corners of a large operator symbol such as $\sum$ or $\prod$,
without affecting the placement of limits.
Examples:
\[\sideset{_*^*}{_*^*}\prod_k\qquad
\sideset{}{'}\sum_{0\le i\le m} E_i\beta x
\]
\begin{verbatim}
\[\sideset{_*^*}{_*^*}\prod_k\qquad
\sideset{}{'}\sum_{0\le i\le m} E_i\beta x
\]
\end{verbatim}

\subsection{The \cn{text} command}
The main use of the command \cn{text} is for words or phrases in a
display:
\[\mathbf{y}=\mathbf{y}'\quad\text{if and only if}\quad
y'_k=\delta_k y_{\tau(k)}\]
\begin{verbatim}
\[\mathbf{y}=\mathbf{y}'\quad\text{if and only if}\quad
y'_k=\delta_k y_{\tau(k)}\]
\end{verbatim}

\subsection{Operator names}
The more common math functions such as $\log$, $\sin$, and $\lim$
have predefined control sequences: \verb=\log=, \verb=\sin=,
\verb=\lim=.
The \pkg{amsmath} package provides \cn{DeclareMathOperator} and
\cn{DeclareMathOperator*}
for producing new function names that will have the
same typographical treatment.
Examples:
\[\norm{f}_\infty=
\esssup_{x\in R^n}\abs{f(x)}\]
\begin{verbatim}
\[\norm{f}_\infty=
\esssup_{x\in R^n}\abs{f(x)}\]
\end{verbatim}
\[\meas_1\{u\in R_+^1\colon f^*(u)>\alpha\}
=\meas_n\{x\in R^n\colon \abs{f(x)}\geq\alpha\}
\quad \forall\alpha>0.\]
\begin{verbatim}
\[\meas_1\{u\in R_+^1\colon f^*(u)>\alpha\}
=\meas_n\{x\in R^n\colon \abs{f(x)}\geq\alpha\}
\quad \forall\alpha>0.\]
\end{verbatim}
\cn{esssup} and \cn{meas} would be defined in the document preamble as
\begin{verbatim}
\DeclareMathOperator*{\esssup}{ess\,sup}
\DeclareMathOperator{\meas}{meas}
\end{verbatim}

The following special operator names are predefined in the \pkg{amsmath}
package: \cn{varlimsup}, \cn{varliminf}, \cn{varinjlim}, and
\cn{varprojlim}. Here's what they look like in use:
\begin{align}
&\varlimsup_{n\rightarrow\infty}
       \mathcal{Q}(u_n,u_n-u^{\#})\le0\\
&\varliminf_{n\rightarrow\infty}
  \left\lvert a_{n+1}\right\rvert/\left\lvert a_n\right\rvert=0\\
&\varinjlim (m_i^\lambda\cdot)^*\le0\\
&\varprojlim_{p\in S(A)}A_p\le0
\end{align}
\begin{verbatim}
\begin{align}
&\varlimsup_{n\rightarrow\infty}
       \mathcal{Q}(u_n,u_n-u^{\#})\le0\\
&\varliminf_{n\rightarrow\infty}
  \left\lvert a_{n+1}\right\rvert/\left\lvert a_n\right\rvert=0\\
&\varinjlim (m_i^\lambda\cdot)^*\le0\\
&\varprojlim_{p\in S(A)}A_p\le0
\end{align}
\end{verbatim}

\subsection{\cn{mod} and its relatives}
The commands \cn{mod} and \cn{pod} are variants of
\cn{pmod} preferred by some authors; \cn{mod} omits the parentheses,
whereas \cn{pod} omits the `mod' and retains the parentheses.
Examples:
\begin{align}
x&\equiv y+1\pmod{m^2}\\
x&\equiv y+1\mod{m^2}\\
x&\equiv y+1\pod{m^2}
\end{align}
\begin{verbatim}
\begin{align}
x&\equiv y+1\pmod{m^2}\\
x&\equiv y+1\mod{m^2}\\
x&\equiv y+1\pod{m^2}
\end{align}
\end{verbatim}

\subsection{Fractions and related constructions}
\label{fracs}

The usual notation for binomials is similar to the fraction concept,
so it has a similar command \cn{binom} with two arguments. Example:
\begin{equation}
\begin{split}
\sum_{\gamma\in\Gamma_C} I_\gamma&
=2^k-\binom{k}{1}2^{k-1}+\binom{k}{2}2^{k-2}\\
&\quad+\dots+(-1)^l\binom{k}{l}2^{k-l}
+\dots+(-1)^k\\
&=(2-1)^k=1
\end{split}
\end{equation}
\begin{verbatim}
\begin{equation}
\begin{split}
[\sum_{\gamma\in\Gamma_C} I_\gamma&
=2^k-\binom{k}{1}2^{k-1}+\binom{k}{2}2^{k-2}\\
&\quad+\dots+(-1)^l\binom{k}{l}2^{k-l}
+\dots+(-1)^k\\
&=(2-1)^k=1
\end{split}
\end{equation}
\end{verbatim}
There are also abbreviations
\begin{verbatim}
\dfrac        \dbinom
\tfrac        \tbinom
\end{verbatim}
for the commonly needed constructions
\begin{verbatim}
{\displaystyle\frac ... }   {\displaystyle\binom ... }
{\textstyle\frac ... }      {\textstyle\binom ... }
\end{verbatim}

The generalized fraction command \cn{genfrac} provides full access to
the six \TeX{} fraction primitives:
\begin{align}
\text{\cn{over}: }&\genfrac{}{}{}{}{n+1}{2}&
\text{\cn{overwithdelims}: }&
  \genfrac{\langle}{\rangle}{}{}{n+1}{2}\\
\text{\cn{atop}: }&\genfrac{}{}{0pt}{}{n+1}{2}&
\text{\cn{atopwithdelims}: }&
  \genfrac{(}{)}{0pt}{}{n+1}{2}\\
\text{\cn{above}: }&\genfrac{}{}{1pt}{}{n+1}{2}&
\text{\cn{abovewithdelims}: }&
  \genfrac{[}{]}{1pt}{}{n+1}{2}
\end{align}
\begin{verbatim}
\text{\cn{over}: }&\genfrac{}{}{}{}{n+1}{2}&
\text{\cn{overwithdelims}: }&
  \genfrac{\langle}{\rangle}{}{}{n+1}{2}\\
\text{\cn{atop}: }&\genfrac{}{}{0pt}{}{n+1}{2}&
\text{\cn{atopwithdelims}: }&
  \genfrac{(}{)}{0pt}{}{n+1}{2}\\
\text{\cn{above}: }&\genfrac{}{}{1pt}{}{n+1}{2}&
\text{\cn{abovewithdelims}: }&
  \genfrac{[}{]}{1pt}{}{n+1}{2}
\end{verbatim}

\subsection{Continued fractions}
The continued fraction
\begin{equation}
\cfrac{1}{\sqrt{2}+
 \cfrac{1}{\sqrt{2}+
  \cfrac{1}{\sqrt{2}+
   \cfrac{1}{\sqrt{2}+
    \cfrac{1}{\sqrt{2}+\dotsb
}}}}}
\end{equation}
can be obtained by typing
\begin{verbatim}
\cfrac{1}{\sqrt{2}+
 \cfrac{1}{\sqrt{2}+
  \cfrac{1}{\sqrt{2}+
   \cfrac{1}{\sqrt{2}+
    \cfrac{1}{\sqrt{2}+\dotsb
}}}}}
\end{verbatim}
Left or right placement of any of the numerators is accomplished by using
\cn{cfrac[l]} or \cn{cfrac[r]} instead of \cn{cfrac}.

\subsection{Smash}

In \pkg{amsmath} there are optional arguments \verb"t" and \verb"b" for
the plain \TeX\ command \cn{smash}, because sometimes it is advantageous
to be able to `smash' only the top or only the bottom of something while
retaining the natural depth or height. In the formula
$X_j=(1/\sqrt{\smash[b]{\lambda_j}})X_j'$ \cn{smash}\verb=[b]= has been
used to limit the size of the radical symbol.
\begin{verbatim}
$X_j=(1/\sqrt{\smash[b]{\lambda_j}})X_j'$
\end{verbatim}
Without the use of \cn{smash}\verb=[b]= the formula would have appeared
thus: $X_j=(1/\sqrt{\lambda_j})X_j'$, with the radical extending to
encompass the depth of the subscript $j$.

\subsection{The `cases' environment}
`Cases' constructions like the following can be produced using
the \env{cases} environment.
\begin{equation}
P_{r-j}=
  \begin{cases}
    0&  \text{if $r-j$ is odd},\\
    r!\,(-1)^{(r-j)/2}&  \text{if $r-j$ is even}.
  \end{cases}
\end{equation}
\begin{verbatim}
\begin{equation} P_{r-j}=
  \begin{cases}
    0&  \text{if $r-j$ is odd},\\
    r!\,(-1)^{(r-j)/2}&  \text{if $r-j$ is even}.
  \end{cases}
\end{equation}
\end{verbatim}
Notice the use of \cn{text} and the embedded math.

\subsection{Matrix}

Here are samples of the matrix environments,
\cn{matrix}, \cn{pmatrix}, \cn{bmatrix}, \cn{Bmatrix}, \cn{vmatrix}
and \cn{Vmatrix}:
\begin{equation}
\begin{matrix}
\vartheta& \varrho\\\varphi& \varpi
\end{matrix}\quad
\begin{pmatrix}
\vartheta& \varrho\\\varphi& \varpi
\end{pmatrix}\quad
\begin{bmatrix}
\vartheta& \varrho\\\varphi& \varpi
\end{bmatrix}\quad
\begin{Bmatrix}
\vartheta& \varrho\\\varphi& \varpi
\end{Bmatrix}\quad
\begin{vmatrix}
\vartheta& \varrho\\\varphi& \varpi
\end{vmatrix}\quad
\begin{Vmatrix}
\vartheta& \varrho\\\varphi& \varpi
\end{Vmatrix}
\end{equation}
%
\begin{verbatim}
\begin{matrix}
\vartheta& \varrho\\\varphi& \varpi
\end{matrix}\quad
\begin{pmatrix}
\vartheta& \varrho\\\varphi& \varpi
\end{pmatrix}\quad
\begin{bmatrix}
\vartheta& \varrho\\\varphi& \varpi
\end{bmatrix}\quad
\begin{Bmatrix}
\vartheta& \varrho\\\varphi& \varpi
\end{Bmatrix}\quad
\begin{vmatrix}
\vartheta& \varrho\\\varphi& \varpi
\end{vmatrix}\quad
\begin{Vmatrix}
\vartheta& \varrho\\\varphi& \varpi
\end{Vmatrix}
\end{verbatim}

To produce a small matrix suitable for use in text, use the
\env{smallmatrix} environment.
\begin{verbatim}
\begin{math}
  \bigl( \begin{smallmatrix}
      a&b\\ c&d
    \end{smallmatrix} \bigr)
\end{math}
\end{verbatim}
To show
the effect of the matrix on the surrounding lines of
a paragraph, we put it here: \begin{math}
  \bigl( \begin{smallmatrix}
      a&b\\ c&d
    \end{smallmatrix} \bigr)
\end{math}
and follow it with enough text to ensure that there will
be at least one full line below the matrix.

\cn{hdotsfor}\verb"{"\textit{number}\verb"}" produces a row of dots in a matrix
spanning the given number of columns:
\[W(\Phi)= \begin{Vmatrix}
\dfrac\varphi{(\varphi_1,\varepsilon_1)}&0&\dots&0\\
\dfrac{\varphi k_{n2}}{(\varphi_2,\varepsilon_1)}&
\dfrac\varphi{(\varphi_2,\varepsilon_2)}&\dots&0\\
\hdotsfor{5}\\
\dfrac{\varphi k_{n1}}{(\varphi_n,\varepsilon_1)}&
\dfrac{\varphi k_{n2}}{(\varphi_n,\varepsilon_2)}&\dots&
\dfrac{\varphi k_{n\,n-1}}{(\varphi_n,\varepsilon_{n-1})}&
\dfrac{\varphi}{(\varphi_n,\varepsilon_n)}
\end{Vmatrix}\]
\begin{verbatim}
\[W(\Phi)= \begin{Vmatrix}
\dfrac\varphi{(\varphi_1,\varepsilon_1)}&0&\dots&0\\
\dfrac{\varphi k_{n2}}{(\varphi_2,\varepsilon_1)}&
\dfrac\varphi{(\varphi_2,\varepsilon_2)}&\dots&0\\
\hdotsfor{5}\\
\dfrac{\varphi k_{n1}}{(\varphi_n,\varepsilon_1)}&
\dfrac{\varphi k_{n2}}{(\varphi_n,\varepsilon_2)}&\dots&
\dfrac{\varphi k_{n\,n-1}}{(\varphi_n,\varepsilon_{n-1})}&
\dfrac{\varphi}{(\varphi_n,\varepsilon_n)}
\end{Vmatrix}\]
\end{verbatim}
The spacing of the dots can be varied through use of a square-bracket
option, for example, \verb"\hdotsfor[1.5]{3}".  The number in square brackets
will be used as a multiplier; the normal value is 1.

\subsection{The \cn{substack} command}

The \cn{substack} command can be used to produce a multiline
subscript or superscript:
for example
\begin{verbatim}
\sum_{\substack{0\le i\le m\\ 0<j<n}} P(i,j)
\end{verbatim}
produces a two-line subscript underneath the sum:
\begin{equation}
\sum_{\substack{0\le i\le m\\ 0<j<n}} P(i,j)
\end{equation}
A slightly more generalized form is the \env{subarray} environment which
allows you to specify that each line should be left-aligned instead of
centered, as here:
\begin{equation}
\sum_{\begin{subarray}{l}
        0\le i\le m\\ 0<j<n
      \end{subarray}}
 P(i,j)
\end{equation}
\begin{verbatim}
\sum_{\begin{subarray}{l}
        0\le i\le m\\ 0<j<n
      \end{subarray}}
 P(i,j)
\end{verbatim}


\subsection{Big-g-g delimiters}
Here are some big delimiters, first in \cn{normalsize}:
\[\biggl(\mathbf{E}_{y}
  \int_0^{t_\varepsilon}L_{x,y^x(s)}\varphi(x)\,ds
  \biggr)
\]
\begin{verbatim}
\[\biggl(\mathbf{E}_{y}
  \int_0^{t_\varepsilon}L_{x,y^x(s)}\varphi(x)\,ds
  \biggr)
\]
\end{verbatim}
and now in \cn{Large} size:
{\Large
\[\biggl(\mathbf{E}_{y}
  \int_0^{t_\varepsilon}L_{x,y^x(s)}\varphi(x)\,ds
  \biggr)
\]}
\begin{verbatim}
{\Large
\[\biggl(\mathbf{E}_{y}
  \int_0^{t_\varepsilon}L_{x,y^x(s)}\varphi(x)\,ds
  \biggr)
\]}
\end{verbatim}

\newpage
%%%%%%%%%%%%%%%%%%%%%%%%%%%%%%%%%%%%%%%%%%%%%%%%%%%%%%%%%%%%%%%%%%%%%%%%%%%
\makeatletter

%% This turns on vertical rules at the right and left margins, to
%% better illustrate the spacing for certain multiple-line equation
%% structures.
\def\@makecol{\ifvoid\footins \setbox\@outputbox\box\@cclv
   \else\setbox\@outputbox
     \vbox{\boxmaxdepth \maxdepth
     \unvbox\@cclv\vskip\skip\footins\footnoterule\unvbox\footins}\fi
  \xdef\@freelist{\@freelist\@midlist}\gdef\@midlist{}\@combinefloats
  \setbox\@outputbox\hbox{\vrule width\marginrulewidth
        \vbox to\@colht{\boxmaxdepth\maxdepth
         \@texttop\dimen128=\dp\@outputbox\unvbox\@outputbox
         \vskip-\dimen128\@textbottom}%
        \vrule width\marginrulewidth}%
     \global\maxdepth\@maxdepth}
\newdimen\marginrulewidth
\setlength{\marginrulewidth}{.1pt}
\makeatother

%%%%%%%%%%%%%%%%%%%%%%%%%%%%%%%%%%%%%%%%%%%%%%%%%%%%%%%%%%%%%%%%%%%%%%%%%%

\appendix
\section{Examples of multiple-line equation structures}
\label{s:eq}

\textbf{\large Note: Starting on this page, vertical rules are
added at the margins so that the positioning of various display elements
with respect to the margins can be seen more clearly.}

\subsection{Split}
The \env{split} environment is not an independent environment
but should be used inside something else such as \env{equation}
or \env{align}.

If there is not enough room for it, the equation number for  a
\env{split} will be shifted to the previous line, when equation numbers are
on the left; the number shifts down to the next line when numbers are on
the right.
\begin{equation}
\begin{split}
f_{h,\varepsilon}(x,y)
&=\varepsilon\mathbf{E}_{x,y}\int_0^{t_\varepsilon}
L_{x,y_\varepsilon(\varepsilon u)}\varphi(x)\,du\\
&= h\int L_{x,z}\varphi(x)\rho_x(dz)\\
&\quad+h\biggl[\frac{1}{t_\varepsilon}\biggl(\mathbf{E}_{y}
  \int_0^{t_\varepsilon}L_{x,y^x(s)}\varphi(x)\,ds
  -t_\varepsilon\int L_{x,z}\varphi(x)\rho_x(dz)\biggr)\\
&\phantom{{=}+h\biggl[}+\frac{1}{t_\varepsilon}
  \biggl(\mathbf{E}_{y}\int_0^{t_\varepsilon}L_{x,y^x(s)}
    \varphi(x)\,ds -\mathbf{E}_{x,y}\int_0^{t_\varepsilon}
   L_{x,y_\varepsilon(\varepsilon s)}
   \varphi(x)\,ds\biggr)\biggr]\\
&=h\wh{L}_x\varphi(x)+h\theta_\varepsilon(x,y),
\end{split}
\end{equation}
Some text after to test the below-display spacing.

\begin{verbatim}
\begin{equation}
\begin{split}
f_{h,\varepsilon}(x,y)
&=\varepsilon\mathbf{E}_{x,y}\int_0^{t_\varepsilon}
L_{x,y_\varepsilon(\varepsilon u)}\varphi(x)\,du\\
&= h\int L_{x,z}\varphi(x)\rho_x(dz)\\
&\quad+h\biggl[\frac{1}{t_\varepsilon}\biggl(\mathbf{E}_{y}
  \int_0^{t_\varepsilon}L_{x,y^x(s)}\varphi(x)\,ds
  -t_\varepsilon\int L_{x,z}\varphi(x)\rho_x(dz)\biggr)\\
&\phantom{{=}+h\biggl[}+\frac{1}{t_\varepsilon}
  \biggl(\mathbf{E}_{y}\int_0^{t_\varepsilon}L_{x,y^x(s)}
    \varphi(x)\,ds -\mathbf{E}_{x,y}\int_0^{t_\varepsilon}
   L_{x,y_\varepsilon(\varepsilon s)}
   \varphi(x)\,ds\biggr)\biggr]\\
&=h\wh{L}_x\varphi(x)+h\theta_\varepsilon(x,y),
\end{split}
\end{equation}
\end{verbatim}
%%%%%%%%%%%%%%%%%%%%%%%%%%%%%%%%%%%%%%%%

\newpage
Unnumbered version:
\begin{equation*}
\begin{split}
f_{h,\varepsilon}(x,y)
&=\varepsilon\mathbf{E}_{x,y}\int_0^{t_\varepsilon}
L_{x,y_\varepsilon(\varepsilon u)}\varphi(x)\,du\\
&= h\int L_{x,z}\varphi(x)\rho_x(dz)\\
&\quad+h\biggl[\frac{1}{t_\varepsilon}\biggl(\mathbf{E}_{y}
  \int_0^{t_\varepsilon}L_{x,y^x(s)}\varphi(x)\,ds
  -t_\varepsilon\int L_{x,z}\varphi(x)\rho_x(dz)\biggr)\\
&\phantom{{=}+h\biggl[}+\frac{1}{t_\varepsilon}
  \biggl(\mathbf{E}_{y}\int_0^{t_\varepsilon}L_{x,y^x(s)}
    \varphi(x)\,ds -\mathbf{E}_{x,y}\int_0^{t_\varepsilon}
   L_{x,y_\varepsilon(\varepsilon s)}
   \varphi(x)\,ds\biggr)\biggr]\\
&=h\wh{L}_x\varphi(x)+h\theta_\varepsilon(x,y),
\end{split}
\end{equation*}
Some text after to test the below-display spacing.

\begin{verbatim}
\begin{equation*}
\begin{split}
f_{h,\varepsilon}(x,y)
&=\varepsilon\mathbf{E}_{x,y}\int_0^{t_\varepsilon}
L_{x,y_\varepsilon(\varepsilon u)}\varphi(x)\,du\\
&= h\int L_{x,z}\varphi(x)\rho_x(dz)\\
&\quad+h\biggl[\frac{1}{t_\varepsilon}\biggl(\mathbf{E}_{y}
  \int_0^{t_\varepsilon}L_{x,y^x(s)}\varphi(x)\,ds
  -t_\varepsilon\int L_{x,z}\varphi(x)\rho_x(dz)\biggr)\\
&\phantom{{=}+h\biggl[}+\frac{1}{t_\varepsilon}
  \biggl(\mathbf{E}_{y}\int_0^{t_\varepsilon}L_{x,y^x(s)}
    \varphi(x)\,ds -\mathbf{E}_{x,y}\int_0^{t_\varepsilon}
   L_{x,y_\varepsilon(\varepsilon s)}
   \varphi(x)\,ds\biggr)\biggr]\\
&=h\wh{L}_x\varphi(x)+h\theta_\varepsilon(x,y),
\end{split}
\end{equation*}
\end{verbatim}
%%%%%%%%%%%%%%%%%%%%%%%%%%%

\newpage
If the option \env{centertags} is included in the options
list of the \pkg{amsmath} package,
the equation numbers for \env{split} environments will be
centered vertically on the height
of  the \env{split}:
{\makeatletter\ctagsplit@true
\begin{equation}
\begin{split}
 \abs{I_2}&=\left\lvert \int_{0}^T \psi(t)\left\{u(a,t)-\int_{\gamma(t)}^a
  \frac{d\theta}{k(\theta,t)}
  \int_{a}^\theta c(\xi)u_t(\xi,t)\,d\xi\right\}dt\right\rvert\\
&\le C_6\left\lvert \left\lvert f\int_\Omega\left\lvert \wt{S}^{-1,0}_{a,-}
  W_2(\Omega,\Gamma_l)\right\rvert\right\rvert
  \left\lvert \abs{u}\overset{\circ}\to W_2^{\wt{A}}
  (\Omega;\Gamma_r,T)\right\rvert\right\rvert.
\end{split}
\end{equation}}%
Some text after to test the below-display spacing.

%%%%%%%%%%%%%%%%%%%%%%%%%%%

\newpage
Use of \env{split} within \env{align}:
{\delimiterfactor750
\begin{align}
\begin{split}\abs{I_1}
  &=\left\lvert \int_\Omega gRu\,d\Omega\right\rvert\\
&\le C_3\left[\int_\Omega\left(\int_{a}^x
  g(\xi,t)\,d\xi\right)^2d\Omega\right]^{1/2}\\
&\quad\times \left[\int_\Omega\left\{u^2_x+\frac{1}{k}
  \left(\int_{a}^x cu_t\,d\xi\right)^2\right\}
  c\Omega\right]^{1/2}\\
&\le C_4\left\lvert \left\lvert f\left\lvert \wt{S}^{-1,0}_{a,-}
  W_2(\Omega,\Gamma_l)\right\rvert\right\rvert
  \left\lvert \abs{u}\overset{\circ}\to W_2^{\wt{A}}
  (\Omega;\Gamma_r,T)\right\rvert\right\rvert.
\end{split}\label{eq:A}\\
\begin{split}\abs{I_2}&=\left\lvert \int_{0}^T \psi(t)\left\{u(a,t)
  -\int_{\gamma(t)}^a\frac{d\theta}{k(\theta,t)}
  \int_{a}^\theta c(\xi)u_t(\xi,t)\,d\xi\right\}dt\right\rvert\\
&\le C_6\left\lvert \left\lvert f\int_\Omega
 \left\lvert \wt{S}^{-1,0}_{a,-}
  W_2(\Omega,\Gamma_l)\right\rvert\right\rvert
  \left\lvert \abs{u}\overset{\circ}\to W_2^{\wt{A}}
  (\Omega;\Gamma_r,T)\right\rvert\right\rvert.
\end{split}
\end{align}}%
Some text after to test the below-display spacing.

\begin{verbatim}
\begin{align}
\begin{split}\abs{I_1}
  &=\left\lvert \int_\Omega gRu\,d\Omega\right\rvert\\
&\le C_3\left[\int_\Omega\left(\int_{a}^x
  g(\xi,t)\,d\xi\right)^2d\Omega\right]^{1/2}\\
&\quad\times \left[\int_\Omega\left\{u^2_x+\frac{1}{k}
  \left(\int_{a}^x cu_t\,d\xi\right)^2\right\}
  c\Omega\right]^{1/2}\\
&\le C_4\left\lvert \left\lvert f\left\lvert \wt{S}^{-1,0}_{a,-}
  W_2(\Omega,\Gamma_l)\right\rvert\right\rvert
  \left\lvert \abs{u}\overset{\circ}\to W_2^{\wt{A}}
  (\Omega;\Gamma_r,T)\right\rvert\right\rvert.
\end{split}\label{eq:A}\\
\begin{split}\abs{I_2}&=\left\lvert \int_{0}^T \psi(t)\left\{u(a,t)
  -\int_{\gamma(t)}^a\frac{d\theta}{k(\theta,t)}
  \int_{a}^\theta c(\xi)u_t(\xi,t)\,d\xi\right\}dt\right\rvert\\
&\le C_6\left\lvert \left\lvert f\int_\Omega
  \left\lvert \wt{S}^{-1,0}_{a,-}
  W_2(\Omega,\Gamma_l)\right\rvert\right\rvert
  \left\lvert \abs{u}\overset{\circ}\to W_2^{\wt{A}}
  (\Omega;\Gamma_r,T)\right\rvert\right\rvert.
\end{split}
\end{align}
\end{verbatim}

%%%%%%%%%%%%%%%%%%

\newpage
Unnumbered \env{align}, with a number on the second \env{split}:
\begin{align*}
\begin{split}\abs{I_1}&=\left\lvert \int_\Omega gRu\,d\Omega\right\rvert\\
  &\le C_3\left[\int_\Omega\left(\int_{a}^x
  g(\xi,t)\,d\xi\right)^2d\Omega\right]^{1/2}\\
&\phantom{=}\times \left[\int_\Omega\left\{u^2_x+\frac{1}{k}
  \left(\int_{a}^x cu_t\,d\xi\right)^2\right\}
  c\Omega\right]^{1/2}\\
&\le C_4\left\lvert \left\lvert f\left\lvert \wt{S}^{-1,0}_{a,-}
  W_2(\Omega,\Gamma_l)\right\rvert\right\rvert
  \left\lvert \abs{u}\overset{\circ}\to W_2^{\wt{A}}
  (\Omega;\Gamma_r,T)\right\rvert\right\rvert.
\end{split}\\
\begin{split}\abs{I_2}&=\left\lvert \int_{0}^T \psi(t)\left\{u(a,t)
  -\int_{\gamma(t)}^a\frac{d\theta}{k(\theta,t)}
  \int_{a}^\theta c(\xi)u_t(\xi,t)\,d\xi\right\}dt\right\rvert\\
&\le C_6\left\lvert \left\lvert f\int_\Omega
  \left\lvert \wt{S}^{-1,0}_{a,-}
  W_2(\Omega,\Gamma_l)\right\rvert\right\rvert
  \left\lvert \abs{u}\overset{\circ}\to W_2^{\wt{A}}
  (\Omega;\Gamma_r,T)\right\rvert\right\rvert.
\end{split}\tag{\theequation$'$}
\end{align*}
Some text after to test the below-display spacing.

\begin{verbatim}
\begin{align*}
\begin{split}\abs{I_1}&=\left\lvert \int_\Omega gRu\,d\Omega\right\rvert\\
  &\le C_3\left[\int_\Omega\left(\int_{a}^x
  g(\xi,t)\,d\xi\right)^2d\Omega\right]^{1/2}\\
&\phantom{=}\times \left[\int_\Omega\left\{u^2_x+\frac{1}{k}
  \left(\int_{a}^x cu_t\,d\xi\right)^2\right\}
  c\Omega\right]^{1/2}\\
&\le C_4\left\lvert \left\lvert f\left\lvert \wt{S}^{-1,0}_{a,-}
  W_2(\Omega,\Gamma_l)\right\rvert\right\rvert
  \left\lvert \abs{u}\overset{\circ}\to W_2^{\wt{A}}
  (\Omega;\Gamma_r,T)\right\rvert\right\rvert.
\end{split}\\
\begin{split}\abs{I_2}&=\left\lvert \int_{0}^T \psi(t)\left\{u(a,t)
  -\int_{\gamma(t)}^a\frac{d\theta}{k(\theta,t)}
  \int_{a}^\theta c(\xi)u_t(\xi,t)\,d\xi\right\}dt\right\rvert\\
&\le C_6\left\lvert \left\lvert f\int_\Omega
  \left\lvert \wt{S}^{-1,0}_{a,-}
  W_2(\Omega,\Gamma_l)\right\rvert\right\rvert
  \left\lvert \abs{u}\overset{\circ}\to W_2^{\wt{A}}
  (\Omega;\Gamma_r,T)\right\rvert\right\rvert.
\end{split}\tag{\theequation$'$}
\end{align*}
\end{verbatim}
%%%%%%%%%%%%%%%%%%%%%%%%%%%

\newpage
\subsection{Multline}
Numbered version:
\begin{multline}\label{eq:E}
\int_a^b\biggl\{\int_a^b[f(x)^2g(y)^2+f(y)^2g(x)^2]
 -2f(x)g(x)f(y)g(y)\,dx\biggr\}\,dy \\
 =\int_a^b\biggl\{g(y)^2\int_a^bf^2+f(y)^2
  \int_a^b g^2-2f(y)g(y)\int_a^b fg\biggr\}\,dy
\end{multline}
To test the use of \verb=\label= and
\verb=\ref=, we refer to the number of this
equation here: (\ref{eq:E}).

\begin{verbatim}
\begin{multline}\label{eq:E}
\int_a^b\biggl\{\int_a^b[f(x)^2g(y)^2+f(y)^2g(x)^2]
 -2f(x)g(x)f(y)g(y)\,dx\biggr\}\,dy \\
 =\int_a^b\biggl\{g(y)^2\int_a^bf^2+f(y)^2
  \int_a^b g^2-2f(y)g(y)\int_a^b fg\biggr\}\,dy
\end{multline}
\end{verbatim}
%%%%%%%%%%%%%%%%%%%%%%%%%%%

Unnumbered version:
\begin{multline*}
\int_a^b\biggl\{\int_a^b[f(x)^2g(y)^2+f(y)^2g(x)^2]
 -2f(x)g(x)f(y)g(y)\,dx\biggr\}\,dy \\
 =\int_a^b\biggl\{g(y)^2\int_a^bf^2+f(y)^2
  \int_a^b g^2-2f(y)g(y)\int_a^b fg\biggr\}\,dy
\end{multline*}
Some text after to test the below-display spacing.

\begin{verbatim}
\begin{multline*}
\int_a^b\biggl\{\int_a^b[f(x)^2g(y)^2+f(y)^2g(x)^2]
 -2f(x)g(x)f(y)g(y)\,dx\biggr\}\,dy \\
 =\int_a^b\biggl\{g(y)^2\int_a^bf^2+f(y)^2
  \int_a^b g^2-2f(y)g(y)\int_a^b fg\biggr\}\,dy
\end{multline*}
\end{verbatim}
%%%%%%%%%%%%%%%%%%%%%%%%%%%

\iffalse % bugfix needed, error message "Multiple \tag"
         % [mjd,24-Jan-1995]
\newpage
And now an ``unnumbered'' version numbered with a literal tag:
\begin{multline*}\tag*{[a]}
\int_a^b\biggl\{\int_a^b[f(x)^2g(y)^2+f(y)^2g(x)^2]
 -2f(x)g(x)f(y)g(y)\,dx\biggr\}\,dy \\
 =\int_a^b\biggl\{g(y)^2\int_a^bf^2+f(y)^2
  \int_a^b g^2-2f(y)g(y)\int_a^b fg\biggr\}\,dy
\end{multline*}
Some text after to test the below-display spacing.

\begin{verbatim}
\begin{multline*}\tag*{[a]}
\int_a^b\biggl\{\int_a^b[f(x)^2g(y)^2+f(y)^2g(x)^2]
 -2f(x)g(x)f(y)g(y)\,dx\biggr\}\,dy \\
 =\int_a^b\biggl\{g(y)^2\int_a^bf^2+f(y)^2
  \int_a^b g^2-2f(y)g(y)\int_a^b fg\biggr\}\,dy
\end{multline*}
\end{verbatim}

%%%%%%%%%%%%%%%%%%%%%%%%%%%

The same display with \verb=\multlinegap= set to zero.
Notice that the space on the left in
the first line does not change, because of the equation number, while
the second line is pushed over to the right margin.
{\setlength{\multlinegap}{0pt}
\begin{multline*}\tag*{[a]}
\int_a^b\biggl\{\int_a^b[f(x)^2g(y)^2+f(y)^2g(x)^2]
 -2f(x)g(x)f(y)g(y)\,dx\biggr\}\,dy \\
 =\int_a^b\biggl\{g(y)^2\int_a^bf^2+f(y)^2
  \int_a^b g^2-2f(y)g(y)\int_a^b fg\biggr\}\,dy
\end{multline*}}%
Some text after to test the below-display spacing.

\begin{verbatim}
{\setlength{\multlinegap}{0pt}
\begin{multline*}\tag*{[a]}
\int_a^b\biggl\{\int_a^b[f(x)^2g(y)^2+f(y)^2g(x)^2]
 -2f(x)g(x)f(y)g(y)\,dx\biggr\}\,dy \\
 =\int_a^b\biggl\{g(y)^2\int_a^bf^2+f(y)^2
  \int_a^b g^2-2f(y)g(y)\int_a^b fg\biggr\}\,dy
\end{multline*}}
\end{verbatim}
%%%%%%%%%%%%%%%%%%%%%%%%%%%
\fi % matches \iffalse above [mjd,24-Jan-1995]

%%%%%%%%%%%%%%%%%%%%%%%%%%%

\newpage
\subsection{Gather}
Numbered version with \verb;\notag; on the second line:
\begin{gather}
D(a,r)\equiv\{z\in\mathbf{C}\colon \abs{z-a}<r\},\\
\seg(a,r)\equiv\{z\in\mathbf{C}\colon
\Im z= \Im a,\ \abs{z-a}<r\},\notag\\
c(e,\theta,r)\equiv\{(x,y)\in\mathbf{C}
\colon \abs{x-e}<y\tan\theta,\ 0<y<r\},\\
C(E,\theta,r)\equiv\bigcup_{e\in E}c(e,\theta,r).
\end{gather}
\begin{verbatim}
\begin{gather}
D(a,r)\equiv\{z\in\mathbf{C}\colon \abs{z-a}<r\},\\
\seg(a,r)\equiv\{z\in\mathbf{C}\colon
\Im z= \Im a,\ \abs{z-a}<r\},\notag\\
c(e,\theta,r)\equiv\{(x,y)\in\mathbf{C}
\colon \abs{x-e}<y\tan\theta,\ 0<y<r\},\\
C(E,\theta,r)\equiv\bigcup_{e\in E}c(e,\theta,r).
\end{gather}
\end{verbatim}
%%%%%%%%%%%%%%%%%%%%%%%%%%%

Unnumbered version.
\begin{gather*}
D(a,r)\equiv\{z\in\mathbf{C}\colon \abs{z-a}<r\},\\
\seg (a,r)\equiv\{z\in\mathbf{C}\colon
\Im z= \Im a,\ \abs{z-a}<r\},\\
c(e,\theta,r)\equiv\{(x,y)\in\mathbf{C}
 \colon \abs{x-e}<y\tan\theta,\ 0<y<r\},\\
C(E,\theta,r)\equiv\bigcup_{e\in E}c(e,\theta,r).
\end{gather*}
Some text after to test the below-display spacing.
\begin{verbatim}
\begin{gather*}
D(a,r)\equiv\{z\in\mathbf{C}\colon \abs{z-a}<r\},\\
\seg (a,r)\equiv\{z\in\mathbf{C}\colon
\Im z= \Im a,\ \abs{z-a}<r\},\\
c(e,\theta,r)\equiv\{(x,y)\in\mathbf{C}
 \colon \abs{x-e}<y\tan\theta,\ 0<y<r\},\\
C(E,\theta,r)\equiv\bigcup_{e\in E}c(e,\theta,r).
\end{gather*}
\end{verbatim}
%%%%%%%%%%%%%%%%%%%%%%%%%%%

\newpage
\subsection{Align}
Numbered version:
\begin{align}
\gamma_x(t)&=(\cos tu+\sin tx,v),\\
\gamma_y(t)&=(u,\cos tv+\sin ty),\\
\gamma_z(t)&=\left(\cos tu+\frac\alpha\beta\sin tv,
  -\frac\beta\alpha\sin tu+\cos tv\right).
\end{align}
Some text after to test the below-display spacing.

\begin{verbatim}
\begin{align}
\gamma_x(t)&=(\cos tu+\sin tx,v),\\
\gamma_y(t)&=(u,\cos tv+\sin ty),\\
\gamma_z(t)&=\left(\cos tu+\frac\alpha\beta\sin tv,
  -\frac\beta\alpha\sin tu+\cos tv\right).
\end{align}
\end{verbatim}
%%%%%%%%%%%%%%%%%%%%%%%%%%%

Unnumbered version:
\begin{align*}
\gamma_x(t)&=(\cos tu+\sin tx,v),\\
\gamma_y(t)&=(u,\cos tv+\sin ty),\\
\gamma_z(t)&=\left(\cos tu+\frac\alpha\beta\sin tv,
  -\frac\beta\alpha\sin tu+\cos tv\right).
\end{align*}
Some text after to test the below-display spacing.

\begin{verbatim}
\begin{align*}
\gamma_x(t)&=(\cos tu+\sin tx,v),\\
\gamma_y(t)&=(u,\cos tv+\sin ty),\\
\gamma_z(t)&=\left(\cos tu+\frac\alpha\beta\sin tv,
  -\frac\beta\alpha\sin tu+\cos tv\right).
\end{align*}
\end{verbatim}
%%%%%%%%%%%%%%%%%%%%%%%%%%%%%%%%%%%%%%%%

A variation:
\begin{align}
x& =y && \text {by (\ref{eq:C})}\\
x'& = y' && \text {by (\ref{eq:D})}\\
x+x' & = y+y' && \text {by Axiom 1.}
\end{align}
Some text after to test the below-display spacing.

\begin{verbatim}
\begin{align}
x& =y && \text {by (\ref{eq:C})}\\
x'& = y' && \text {by (\ref{eq:D})}\\
x+x' & = y+y' && \text {by Axiom 1.}
\end{align}
\end{verbatim}
%%%%%%%%%%%%%%%%%%%%%%%%%%%

\newpage
\subsection{Align and split within gather}
When using the \env{align} environment within the \env{gather}
environment, one or the other, or both, should be unnumbered (using the
\verb"*" form); numbering both the outer and inner environment would
cause a conflict.

Automatically numbered \env{gather} with \env{split} and \env{align*}:
\begin{gather}
\begin{split} \varphi(x,z)
&=z-\gamma_{10}x-\gamma_{mn}x^mz^n\\
&=z-Mr^{-1}x-Mr^{-(m+n)}x^mz^n
\end{split}\\[6pt]
\begin{align*}
\zeta^0 &=(\xi^0)^2,\\
\zeta^1 &=\xi^0\xi^1,\\
\zeta^2 &=(\xi^1)^2,
\end{align*}
\end{gather}
Here the \env{split} environment gets a number from the outer
\env{gather} environment; numbers for individual lines of the
\env{align*} are suppressed because of the star.

\begin{verbatim}
\begin{gather}
\begin{split} \varphi(x,z)
&=z-\gamma_{10}x-\gamma_{mn}x^mz^n\\
&=z-Mr^{-1}x-Mr^{-(m+n)}x^mz^n
\end{split}\\[6pt]
\begin{align*}
\zeta^0 &=(\xi^0)^2,\\
\zeta^1 &=\xi^0\xi^1,\\
\zeta^2 &=(\xi^1)^2,
\end{align*}
\end{gather}
\end{verbatim}
%%%%%%%%%%%%%%%%%%%%%%%%%%%%%%%%%%%%%%%%%%%%%%%%%%%%%

The \verb"*"-ed form of \env{gather} with the non-\verb"*"-ed form of
\env{align}.
\begin{gather*}
\begin{split} \varphi(x,z)
&=z-\gamma_{10}x-\gamma_{mn}x^mz^n\\
&=z-Mr^{-1}x-Mr^{-(m+n)}x^mz^n
\end{split}\\[6pt]
\begin{align} \zeta^0&=(\xi^0)^2,\\
\zeta^1 &=\xi^0\xi^1,\\
\zeta^2 &=(\xi^1)^2,
\end{align}
\end{gather*}
Some text after to test the below-display spacing.

\begin{verbatim}
\begin{gather*}
\begin{split} \varphi(x,z)
&=z-\gamma_{10}x-\gamma_{mn}x^mz^n\\
&=z-Mr^{-1}x-Mr^{-(m+n)}x^mz^n
\end{split}\\[6pt]
\begin{align} \zeta^0&=(\xi^0)^2,\\
\zeta^1 &=\xi^0\xi^1,\\
\zeta^2 &=(\xi^1)^2,
\end{align}
\end{gather*}
\end{verbatim}
%%%%%%%%%%%%%%%%%%%%%%%%%%%%%%%%%%%%%%%%

\newpage
\subsection{Alignat}
Numbered version:
\begin{alignat}{3}
V_i & =v_i - q_i v_j, & \qquad X_i & = x_i - q_i x_j,
 & \qquad U_i & = u_i,
 \qquad \text{for $i\ne j$;}\label{eq:B}\\
V_j & = v_j, & \qquad X_j & = x_j,
  & \qquad U_j & u_j + \sum_{i\ne j} q_i u_i.
\end{alignat}
Some text after to test the below-display spacing.

\begin{verbatim}
\begin{alignat}{3}
V_i & =v_i - q_i v_j, & \qquad X_i & = x_i - q_i x_j,
 & \qquad U_i & = u_i,
 \qquad \text{for $i\ne j$;}\label{eq:B}\\
V_j & = v_j, & \qquad X_j & = x_j,
  & \qquad U_j & u_j + \sum_{i\ne j} q_i u_i.
\end{alignat}
\end{verbatim}
%%%%%%%%%%%%%%%%%%%%%%%%%%%

Unnumbered version:
\begin{alignat*}3
V_i & =v_i - q_i v_j, & \qquad X_i & = x_i - q_i x_j,
 & \qquad U_i & = u_i,
 \qquad \text{for $i\ne j$;} \\
V_j & = v_j, & \qquad X_j & = x_j,
  & \qquad U_j & u_j + \sum_{i\ne j} q_i u_i.
\end{alignat*}
Some text after to test the below-display spacing.

\begin{verbatim}
\begin{alignat*}3
V_i & =v_i - q_i v_j, & \qquad X_i & = x_i - q_i x_j,
 & \qquad U_i & = u_i,
 \qquad \text{for $i\ne j$;} \\
V_j & = v_j, & \qquad X_j & = x_j,
  & \qquad U_j & u_j + \sum_{i\ne j} q_i u_i.
\end{alignat*}
\end{verbatim}
%%%%%%%%%%%%%%%%%%%%%%%%%%%

\newpage
The most common use for \env{alignat} is for things like
\begin{alignat}{2}
x& =y && \qquad \text {by (\ref{eq:A})}\label{eq:C}\\
x'& = y' && \qquad \text {by (\ref{eq:B})}\label{eq:D}\\
x+x' & = y+y' && \qquad \text {by Axiom 1.}
\end{alignat}
Some text after to test the below-display spacing.

\begin{verbatim}
\begin{alignat}{2}
x& =y && \qquad \text {by (\ref{eq:A})}\label{eq:C}\\
x'& = y' && \qquad \text {by (\ref{eq:B})}\label{eq:D}\\
x+x' & = y+y' && \qquad \text {by Axiom 1.}
\end{alignat}
\end{verbatim}
%%%%%%%%%%%%%%%%%%%%%%%%%%%

\newpage
\setlength{\marginrulewidth}{0pt}

\begin{thebibliography}{10}

\bibitem{dihe:newdir}
W.~Diffie and E.~Hellman, \emph{New directions in cryptography}, IEEE
Transactions on Information Theory \textbf{22} (1976), no.~5, 644--654.

\bibitem{fre:cichon}
D.~H. Fremlin, \emph{Cichon's diagram}, 1983/1984, presented at the
S{\'e}minaire Initiation {\`a} l'Analyse, G. Choquet, M. Rogalski, J.
Saint Raymond, at the Universit{\'e} Pierre et Marie Curie, Paris, 23e
ann{\'e}e.

\bibitem{gouja:lagrmeth}
I.~P. Goulden and D.~M. Jackson, \emph{The enumeration of directed
closed {E}uler trails and directed {H}amiltonian circuits by
{L}angrangian methods}, European J. Combin. \textbf{2} (1981), 131--212.

\bibitem{hapa:graphenum}
F.~Harary and E.~M. Palmer, \emph{Graphical enumeration}, Academic
Press, 1973.

\bibitem{imlelu:oneway}
R.~Impagliazzo, L.~Levin, and M.~Luby, \emph{Pseudo-random generation
from one-way functions}, Proc. 21st STOC (1989), ACM, New York,
pp.~12--24.

\bibitem{komiyo:unipfunc}
M.~Kojima, S.~Mizuno, and A.~Yoshise, \emph{A new continuation method
for complementarity problems with uniform p-functions}, Tech. Report
B-194, Tokyo Inst. of Technology, Tokyo, 1987, Dept. of Information
Sciences.

\bibitem{komiyo:lincomp}
\bysame, \emph{A polynomial-time algorithm for a class of linear
complementarity problems}, Tech. Report B-193, Tokyo Inst. of
Technology, Tokyo, 1987, Dept. of Information Sciences.

\bibitem{liuchow:formalsum}
C.~J. Liu and Yutze Chow, \emph{On operator and formal sum methods for
graph enumeration problems}, SIAM J. Algorithms Discrete Methods
\textbf{5} (1984), 384--438.

\bibitem{mami:matrixth}
M.~Marcus and H.~Minc, \emph{A survey of matrix theory and matrix
inequalities}, Complementary Series in Math. \textbf{14} (1964), 21--48.

\bibitem{miyoki:lincomp}
S.~Mizuno, A.~Yoshise, and T.~Kikuchi, \emph{Practical polynomial time
algorithms for linear complementarity problems}, Tech. Report~13, Tokyo
Inst. of Technology, Tokyo, April 1988, Dept. of Industrial Engineering
and Management.

\bibitem{moad:quadpro}
R.~D. Monteiro and I.~Adler, \emph{Interior path following primal-dual
algorithms, part {II}: Quadratic programming}, August 1987, Working
paper, Dept. of Industrial Engineering and Operations Research.

\bibitem{ste:sint}
E.~M. Stein, \emph{Singular integrals and differentiability properties
of functions}, Princeton Univ. Press, Princeton, N.J., 1970.

\bibitem{ye:intalg}
Y.~Ye, \emph{Interior algorithms for linear, quadratic and linearly
constrained convex programming}, Ph.D. thesis, Stanford Univ., Palo
Alto, Calif., July 1987, Dept. of Engineering--Economic Systems,
unpublished.

\end{thebibliography}

\end{document}
\endinput
